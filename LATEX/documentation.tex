\documentclass[12pt, a4paper]{report}

% --- Základní nastavení kódování a jazyka ---
\usepackage[utf8]{inputenc}
\usepackage[T1]{fontenc}
\usepackage[czech]{babel}

% --- Balíčky pro sazbu ---
\usepackage{graphicx} % Pro obrázky
\usepackage{amsmath} % Pro matematiku
\usepackage{geometry} % Pro nastavení okrajů
\geometry{
 left=30mm,
 right=25mm,
 top=30mm,
 bottom=30mm,
}

% --- Balíčky pro kód a odkazy ---
\usepackage{listings} % Pro výpisy kódu
\usepackage{xcolor} % Pro barvy v kódu
\usepackage{hyperref} % Pro klikatelné odkazy
\hypersetup{
    colorlinks=true,
    linkcolor=blue,
    filecolor=magenta,      
    urlcolor=cyan,
    pdftitle={Dokumentace LOTR Systému},
    pdfpagemode=FullScreen,
}

% --- Nastavení pro výpisy kódu ---
\definecolor{codegreen}{rgb}{0,0.6,0}
\definecolor{codegray}{rgb}{0.5,0.5,0.5}
\definecolor{codepurple}{rgb}{0.58,0,0.82}
\definecolor{backcolour}{rgb}{0.95,0.95,0.92}

\lstdefinestyle{mystyle}{
    backgroundcolor=\color{backcolour},   
    commentstyle=\color{codegreen},
    keywordstyle=\color{magenta},
    numberstyle=\tiny\color{codegray},
    stringstyle=\color{codepurple},
    basicstyle=\ttfamily\footnotesize,
    breakatwhitespace=false,         
    breaklines=true,                 
    captionpos=b,                    
    keepspaces=true,                 
    numbers=left,                    
    numbersep=5pt,                  
    showspaces=false,                
    showstringspaces=false,
    showtabs=false,                  
    tabsize=2
}
\lstset{style=mystyle}


% --- Informace o dokumentu ---
\title{Technická dokumentace \\ GPS Tracking Systém "LOTR"}
\author{Tým LOTR}
\date{\today}


% --- Začátek dokumentu ---
\begin{document}

% --- Titulní strana a obsah ---
\maketitle

% --- Prohlášení a poděkování ---
\cleardoublepage
\pagenumbering{roman}

\chapter*{Prohlášení}
\addcontentsline{toc}{chapter}{Prohlášení}
Prohlašujeme, že tento dokument byl vypracován samostatně a všechny použité zdroje jsou řádně citovány.

\vspace*{2cm}
\noindent V Praze dne \today \hfill \textit{Tým LOTR}

\chapter*{Poděkování}
\addcontentsline{toc}{chapter}{Poděkování}
Děkujeme všem, kteří se podíleli na tomto projektu.

\begin{abstract}
\addcontentsline{toc}{chapter}{Abstrakt}
Tento dokument poskytuje komplexní technický přehled systému pro sledování GPS zařízení "LOTR". Popisuje architekturu backendu, hardwarových komponent a mobilní aplikace.
\end{abstract}

\tableofcontents
\listoffigures
\listoftables

\cleardoublepage
\pagenumbering{arabic}

% --- Hlavní část dokumentu ---


\part{Analytická část}
\chapter{Úvod do problematiky}
\section{Cíle projektu}
\section{Analýza existujících řešení}
\section{Technologický stack}

\chapter{Architektura systému}
\section{Celkový přehled}
\section{Komunikační diagramy}


\part{Návrhová a implementační část}
\chapter{Backend Server (Node.js)}
\section{Přehled a technologie}
\section{Databázové schéma (ERD)}
\section{API a routy}
\section{Zpracování GPS dat}
\section{Autentizace a autorizace}


\chapter{Hardware (GPS Tracker)}
\section{Popis komponent}
\section{Schéma zapojení}
\section{Firmware a provozní režimy}

\chapter{Mobilní aplikace (Android)}
\section{Architektura a použité knihovny}
\section{Uživatelské rozhraní a funkce}

\chapter{Testování}
\section{Unit testy}
\section{Integrační testy}
\section{Uživatelské testování}

\chapter{Závěr}
\section{Shrnutí výsledků}
\section{Možnosti dalšího rozvoje}


% --- Přílohy ---
\appendix
\chapter{Příklady API volání}
\chapter{Výpisy zdrojových kódů}
\chapter{Obsah přiloženého CD/úložiště}

% --- Bibliografie ---
\cleardoublepage
\addcontentsline{toc}{chapter}{Literatura}
\bibliographystyle{czechiso} % Styl citací podle normy
%\bibliography{zdroje} % Název vašeho .bib souboru

\end{document}
