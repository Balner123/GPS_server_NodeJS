% STRUKTURA Z ŠABLONY PRO PSANÍ ZÁVĚREČNÉ STUDIJNÍ PRÁCE
%%%%%%%%%%%%%%%%%%%%%%%%%%%%%%%%%%%%%%%%%%%%
% Autor: Jakub Dokulil (kubadokulil99@gmail.com)

\documentclass[12pt, a4paper,
%oneside,      %% -- odkomentujte, pokud chcete svou práci mít pouze jednostrannou, mezera pro hřbet pak automaticky bude pouze na levé straně
twoside,        %% -- pro oboustranné práce, mezera pro hřbet následně střídá strany.
openright
]{report}

%% Nutné balíčky a nastavení
%%%%%%%%%%%%%%%%%%%%%%%%%%%%

%% Proměnné
\newcommand\obor{INFORMAČNÍ TECHNOLOGIE} %% -- napiš číslo a název tvého oboru
\newcommand\kodOboru{18-20-M/01} %% -- napiš číslo a název tvého oboru
\newcommand\zamereni{se zaměřením na počítačové sítě a programování} %% -- napiš číslo a název tvého oboru
\newcommand\skola{Střední škola průmyslová a umělecká, Opava} %% vyplň název školy
\newcommand\trida{IT4} %% vyplň jméno svého konzultanta
\newcommand\jmenoAutora{Štěpán Balner}  %% vyplň své jméno
\newcommand\skolniRok{2025/26} %% vyplň rok
\newcommand\datumOdevzdani{1. 1. 2026} %% vyplň rok
\newcommand\nazevPrace{LOTR - LOcation TRacking System} %% vyplň název své práce

\title{\nazevPrace} %% -- Název tvé práce
\author{\jmenoAutora} %% -- tvé jméno
\date{\datumOdevzdani} %% -- rok, kdy píšeš SOČku

\usepackage[top=2.5cm, bottom=2.5cm, left=3.5cm, right=1.5cm]{geometry} %% nastaví okraje, left -- vnitřní okraj, right -- vnější okraj

\usepackage[czech]{babel} %% balík babel pro sazbu v češtině
\usepackage[utf8]{inputenc} %% balíky pro kódování textu
\usepackage[T1]{fontenc}
\usepackage{cmap} %% balíček zajišťující, že vytvořené PDF bude prohledávatelné a kopírovatelné

\usepackage{graphicx} %% balík pro vkládání obrázků

\usepackage{subcaption} %% balíček pro vkládání podobrázků
\usepackage{float} %% balíček pro přesné umístění obrázků [H]

\usepackage{hyperref} %% balíček, který v PDF vytváří odkazy

\linespread{1.25} %% řádkování
\setlength{\parskip}{0.5em} %% odsazení mezi odstavci


\usepackage[pagestyles]{titlesec} %% balíček pro úpravu stylu kapitol a sekcí
\titleformat{\chapter}[block]{\scshape\bfseries\LARGE}{\thechapter}{10pt}{\vspace{0pt}}[\vspace{-22pt}]
\titleformat{\section}[block]{\scshape\bfseries\Large}{\thesection}{10pt}{\vspace{0pt}}
\titleformat{\subsection}[block]{\bfseries\large}{\thesubsection}{10pt}{\vspace{0pt}}


\usepackage{tocloft} % Balíček umožní přizpůsobit vzhled tabulky obsahu
\setlength{\cftbeforechapskip}{0pt}  % Menší rozestup pro kapitoly
\setlength{\cftbeforesecskip}{0pt}   % Menší rozestup pro sekce

% Nastavení hloubky číslování a obsahu
\setcounter{secnumdepth}{2} % Číslování až do úrovně 1.1.1
\setcounter{tocdepth}{1}    % Obsah zobrazuje až do úrovně 1.1 (Kapitoly a Sekce)

% Nastavení vzhledu položek v obsahu
\renewcommand{\cftchapfont}{\scshape\bfseries} % Kapitoly: Kapitálky + Tučně
\renewcommand{\cftsecfont}{\scshape}           % Sekce: Kapitálky
\renewcommand{\cftsubsecfont}{\scshape}        % Podsekce: Kapitálky
\renewcommand{\cftsubsubsecfont}{\normalfont}  % Pod-podsekce: Normální písmo (malé)

% Zmenšení mezer pro nejnižší úroveň
\setlength{\cftbeforesubsubsecskip}{-2pt}      % Zmenšená mezera před subsubsection

\usepackage{fancyhdr}
\pagestyle{fancy}
\renewcommand{\headrulewidth}{0.025pt}

\usepackage{booktabs}

\usepackage{url}

%% Balíčky co se můžou hodit :) 
%%%%%%%%%%%%%%%%%%%%%%%%%%%%%%%

\usepackage{pdfpages} %% Balíček umožňující vkládat stránky z PDF souborů, 

\usepackage{upgreek} %% Balíček pro sazbu stojatých řeckých písmen, třeba u jednotky mikrometr. Například stojaté mí: \upmu, stojaté pí: \uppi

\usepackage{amsmath}    %% Balíčky amsmath a amsfonts 
\usepackage{amsfonts}   %% pro sazbu matematických symbolů
\usepackage{esint}     %% pro sazbu různých integrálů (např \oiint)
\usepackage{mathrsfs}
\usepackage{helvet} % Helvet font
\usepackage{mathptmx} % Times New Roman
\usepackage{Oswald} % Oswald font


%% makra pro sazbu matematiky
\newcommand{\dif}{\mathrm{d}} %% makro pro sazbu diferenciálu, místo toho
%% abych musel psát '\mathrm{d}' mi stačí napsat '\dif' což je mnohem 
%% kratší a mohu si tak usnadnit práci

\usepackage{listings}
\usepackage{xcolor}

\renewcommand{\lstlistingname}{Kód}% Listing -> Algorithm
\renewcommand{\lstlistlistingname}{Seznam programových kódů}% List of Listings -> List of Algorithms

\lstalias[]{ES6}[ECMAScript2015]{JavaScript}

% Nastavení barev
% Requires package: color.
\definecolor{mediumgray}{rgb}{0.3, 0.4, 0.4}
\definecolor{mediumblue}{rgb}{0.0, 0.0, 0.8}
\definecolor{forestgreen}{rgb}{0.13, 0.55, 0.13}
\definecolor{darkviolet}{rgb}{0.58, 0.0, 0.83}
\definecolor{royalblue}{rgb}{0.25, 0.41, 0.88}
\definecolor{crimson}{rgb}{0.86, 0.8, 0.24}


\lstdefinestyle{JSES6Base}{
	backgroundcolor=\color{white},
	basicstyle=\ttfamily,
	breakatwhitespace=false,
	breaklines=false,
	captionpos=b,
	columns=fullflexible,
	commentstyle=\color{mediumgray}\upshape,
	emph={},
	emphstyle=\color{crimson},
	extendedchars=true,  % requires inputenc
	fontadjust=true,
	frame=single,
	identifierstyle=\color{black},
	keepspaces=true,
	keywordstyle=\color{mediumblue},
	keywordstyle={[2]\color{darkviolet}},
	keywordstyle={[3]\color{royalblue}},
 literate=%
{á}{{\'a}}1 {č}{{\v{c}}}1 {ď}{{\v{d}}}1 {é}{{\'e}}1 {ě}{{\v{e}}}1
{í}{{\'i}}1 {ň}{{\v{n}}}1 {ó}{{\'o}}1 {ř}{{\v{r}}}1 {š}{{\v{s}}}1
{ť}{{\v{t}}}1 {ú}{{\'u}}1 {ů}{{\r{u}}}1 {ý}{{\'y}}1 {ž}{{\v{z}}}1,		
	numbers=left,
	numbersep=5pt,
	numberstyle=\tiny\color{black},
	rulecolor=\color{black},
	showlines=true,
	showspaces=false,
	showstringspaces=false,
	showtabs=false,
	stringstyle=\color{forestgreen},
	tabsize=2,
	title=\lstname,
	upquote=true  % requires textcomp
}

\lstdefinestyle{JavaScript}{
	language=JavaScript,
	style=JSES6Base,
}
\lstdefinestyle{ES6}{
	language=ES6,
	style=JSES6Base
}

%% Začátek dokumentu
%%%%%%%%%%%%%%%%%%%%
\begin{document}
	
	\pagestyle{empty}
	\pagenumbering{Roman}
	
	\cleardoublepage

%% Titulní stránka s informacemi
%%%%%%%%%%%%%%%%%%%%%%%%%%%%%%%%%%%%%%%%
	
	{\fontfamily{phv}\selectfont
		%% Logo školy
		\begin{figure}[h]
			\centering
			\includegraphics[width=0.6\linewidth]{image/logo-skoly.png} 
		\end{figure}
		
		
		%% Hlavička práce a její název (viz proměnná \nazev prace)
		%% \sffamily %%% bezpatkové písmo - sans serif
		{\bfseries %%% písmo na stránce je tučně
			\begin{center}
				\vspace{0.025 \textheight}
				\LARGE{ZÁVĚREČNÁ STUDIJNÍ PRÁCE}\\
				\large{dokumentace}\\
				\vspace{0.075 \textheight}
				\LARGE {\nazevPrace}\\
			\end{center}  
		}%%%
		
		\begin{figure}[h]
			\centering
			\includegraphics[width=0.4\linewidth]{image/logo_lotr.png} 
		\end{figure}
		
		\vspace{0.02 \textheight}
		\begin{table}[h!]
			\begin{tabular}{ll}
				\textbf{Autor:} & \jmenoAutora\\ 
				\textbf{Obor:} & \kodOboru { } \obor\\
				\textbf{} & \zamereni\\
				\textbf{Třída:} & \trida\\
				\textbf{Školní rok:} & \skolniRok\\
			\end{tabular}
			
		\end{table}		
	}
	
\cleardoublepage %% Zalomení dvojstránky
	
%% Stránka obsahující poděkování a prohlášení
%%%%%%%%%%%%%%%%%%%%%%%%%%%%%%%%%%%%%%%%%%%%%%%%%%%%%%%%

%% Poděkování - nepovinné
%%%%%%%%%%%%%%%%%%%%%%%%%%%%
	
	\noindent{\large{\bfseries{Poděkování}\\}}
	\noindent Děkuji těm, jimž poděkováno zatím nebylo. Také děkuji Současné době za to, že jsou již LLM na takové urovni, že vytvoření tohoto projektu bylo možné i za sníženého pracovního nasazení.\\
	
	\vspace*{0.5\textheight} %% Vertikální mezeru je možné upravit

%% Prohlášení - povinné
%%%%%%%%%%%%%%%%%%%%%%%%%%%%
	\noindent{\large{\bfseries{Prohlášení}\\}}  %% uprav si koncovky podle toho na jaký rod se cítíš, vypadá to pak lépe :) 
	\noindent{Prohlašuji, že jsem závěrečnou práci vypracoval samostatně a uvedl veškeré použité 
		informační zdroje.\\}
	\noindent{Souhlasím, aby tato studijní práce byla použita k výukovým a prezentačním účelům na Střední průmyslové a umělecké škole v Opavě, Praskova 399/8.}
	\vfill
	\noindent{V Opavě \datumOdevzdani\\}
	\noindent
	\begin{minipage}{\linewidth}
		\hspace{9.5cm} 
		\begin{tabular}{@{}p{6cm}@{}}
			\dotfill \\
			Podpis autora
		\end{tabular}
	\end{minipage}

	\cleardoublepage %% Zalomení dvojstránky

%% Stránka obsahující abstrakt (anotaci)
%%%%%%%%%%%%%%%%%%%%%%%%%%%%%%%%%%%%%%%%%%%%%%%%%%%%%%%%	

%% Abstrakt v češtině
%%%%%%%%%%%%%%%%%%%%%%%%%%%%
	\noindent{\Large{\bfseries{Anotace}\\}}
	\noindent Cílem této práce je návrh a realizace sledovacího systému LOTR (LOcation TRacking System), který se skládá z fyzického hardwarového zařízení, serverové infrastruktury a klientské mobilní aplikace.
	
	Hardwarová část je postavena na platformě ESP32 s využitím modemu SIMCOM A7670 pro komunikaci přes sítě LTE a modulu GPS pro získávání polohy. Firmware je optimalizován pro nízkou spotřebu energie a spolehlivý přenos dat. Serverová část, realizovaná v prostředí Node.js s databází MySQL, zajišťuje sběr telemetrických dat, správu uživatelů a poskytuje REST API pro komunikaci s koncovými zařízeními. Součástí systému je také nativní aplikace pro OS Android vyvinutá v jazyce Kotlin, která slouží jako plnohodnotná softwarová náhrada hardwarového trackeru.
	
	Výsledkem projektu je prototyp IoT řešení umožňující sledování polohy v reálném čase, historii pohybu a vzdálenou konfiguraci připojených zařízení.
	
	
	\vspace{18pt}
	
	\noindent{\large{\bfseries{Klíčová slova}}}
	
	\noindent GPS, IoT, sledovací systém, ESP32, LTE, Node.js, MySQL, Android, Kotlin, REST API
	
	\vspace{18pt}
	
	\clearpage %% Zalomení stránky

%% Stránka s generovaným obsahem
%%%%%%%%%%%%%%%%%%%%%%%%%%%%%%%%%%%%%%%	
	
	\tableofcontents %% Vygeneruje tabulku s obsahem

	\pagenumbering{arabic} %% Nastavení způsobu číslování stránek (alternativy roman | Roman)
	\setcounter{page}{1} %% Nastavení počitadla stránek

%% Stránka s úvodem - povinná část
%%%%%%%%%%%%%%%%%%%%%%%%%%%%%%%%%%%%%%%		
	\chapter*{Úvod}
	Můžeme říci, že již od začátku cílem projektu bylo vytvořit kompletní a hlavně uzavřený systém, co by mohl žít svým vlastním životem a jehož použití by bylo možné nasadit v reálných podmínkách.
	
	Již kdysi jsem se zabýval "sběrnými" nebo "sledovacími" systémy. Poohlížel jsem se z variabilních důvodů po možnostech jak by šlo sestrojit zařízení schopné nahrávat video a ukládat na SD kartu nebo jiné uložiště.
	Sledování polohy je jenom dalším krokem v tomto směru. Možnost že bych měl krabičku jež bych mohl dát někomu do auta, nebo jej připnout k nápravě kamionu a sledoval jeho pohyb do doby vybití baterie byla, a stále je, velmi přitažlivá.
	
	Bylo jasné, že již existují řešení a "GPS-trackery" jsou běžně dostupné a velmi dobré kvality a výdrže baterie. Avšak protože jsem dumal nad tématem pro zavěrečný projekt, výroba a řešení konstrukce vlastního, sic nízkokvalitního a dosti poruchového, zařízení se ukázal jako vhodný nápad.
	Existovala možnost vytvořit opravdu pouze "krabici" trackeru s ukládáním dat n apevné úložiště , avšak rozhodl jsem se to spojit s vývojem serveru jež by umožňoval sledování v reálném čase. Na tomto "webovém" serveru by tak šlo zobrazovat a spravovat data posílaná  zařízením přes mobilní síť.
	To se pak přirozeně rozvinulo do plného systému s uživatelskými účty a možností spravovat více zařízení na jednom z nich. 
	Již z počátků pak bylo také na výledni ,že pokud by server měl fungovat a přijímat data spolehlivě, tak jeho jádrem by muselo být responzivní a hlavně "robustní" API. Jeho struktura se však, hlavně kvůli postupnému přehodnocuvání principu komunikace mezi serverem a zařízeními, dosti drasticky měnila.
	Fyzické zařízení totiž bylo hlavním pilířem celého systému a jeho kontrukce, složení, a hlavně samotný kod, udávaly směr vývoje všeho ostatního. 

	Jako méně důležitou, spíše doplňkovou, ale přesto užitečnou součást systému jsem pak viděl souběžný vývoj mobilní aplikace pro systém Android. Ta měla sloužit jako plnohodnotná náhrada hardwarového zařízení a možná jako náhrada více schopná.

%Příkaz \chapter*{Úvod}	
%Tento příkaz vytvoří novou kapitolu s názvem "Úvod" ve vašem dokumentu.
%Hvězdička * u příkazu \chapter* znamená, že tato kapitola nebude mít číslo. Ve výsledném dokumentu se tedy objeví jako "Úvod" bez předcházejícího čísla kapitoly, které se obvykle zobrazuje u číslovaných kapitol.
%Tento příkaz také znamená, že kapitola se automaticky neobjeví v obsahu, protože LaTeX standardně zahrnuje do obsahu pouze číslované kapitoly.
	\addcontentsline{toc}{chapter}{Úvod}
%Tento příkaz ručně přidává záznam do obsahu.
%První parametr toc označuje, že přidáváme záznam do Table of Contents (obsahu).
%Druhý parametr chapter specifikuje úroveň záznamu. V tomto případě říkáme, že přidávaný záznam má být považován za kapitolu.
%Třetí parametr Úvod je text, který se objeví v obsahu. V tomto případě bude v obsahu zobrazen název "Úvod".	

%Tipy k psaní úvodu
%Je povinný, nadpis neměňte, rozsah - max. 1 strana. 
%Tato část práce obsahuje: 
%* náhled do řešené problematiky, zdůvodnění volby problematiky, 
%* předem definované cíle práce, 
%* motivaci pro další čtení textu včetně stručného uvedení obsahu následujících kapitol 

\chapter{Fyzické zařízení - tracker}
\label{chap:hw_tracker}

\section{Úvod a koncepce}
Tracker je základní součást systému, vše ostaní se zrodilo z jeho potřeb o rozšíření, proto je to hlavní část projektu. Jeho hlavním úkolem je pravidelně zaznamenávat polohu a odesílat ji na server, dle požadavků a jednotlivých nastavení uživatelem. 
Hlavními (základními) požadavky na zařízení jsou:

\begin{itemize}
	\item Zisk GPS dat
	\iten Jejich odeslání na Server
\end{itemize}

\noindent To jsou však pouze základní požadavky, se kterými návrh začínal. Postup času se poté rozvinul v mnohem komplexní systém chování (např. cachovaní, deepslep).
Zde přikládám digram použítí pro finální konfiguraci zařízení.

\begin{figure}[h]
    \centering
    \includegraphics[width=0.9\linewidth]{../docs_hw/schemas/use_case_dia.png} 
    \caption{Usecase diagram}
    \label{fig:use_case_dia}
\end{figure}

\section{POUŽITÉ TECHNOLOGIE}
\label{sec:hw_teorie}

\subsection{Mikrokontroléry (ESP32)}
Jako řídící jednotka byl zvolen čip \textbf{ESP32} od společnosti Espressif Systems. Jedná se o 32-bitový mikrokontrolér (architektura Xtensa LX6) s integrovanou Wi-Fi a Bluetooth konektivitou. 
Pro tento projekt je klíčová jeho pokročilá správa napájení, konkrétně režim \textit{Deep Sleep} viz \ref{sec:deep_sleep}.

\subsection{Mobilní komunikace a AT příkazy}
Pro přenos dat využíváme síť LTE (4G). Komunikace mezi mikrokontrolérem a modemem probíhá po sériové lince (UART) pomocí standardizované sady \textbf{AT příkazů} (Hayes command set).

\subsection{Globální navigační systémy (GPS)}
Pro získávání polohy je využit systém GPS. Modul komunikuje s mikrokontrolérem pomocí textového protokolu \textbf{NMEA}. 
Z celého proudu dat jsou parsovány především věty \texttt{\$GPRMC} (Recommended Minimum Specific GPS/TRANSIT Data), 
které obsahují klíčové údaje o poloze, času a rychlosti.

\subsection{Operační systémy reálného času (FreeRTOS)}
Pro zajištění deterministického chování a správy souběžných procesů (komunikace, sběr dat) je využit operační systém reálného času \textbf{FreeRTOS}. 
Ten umožňuje rozdělit aplikaci do samostatných úloh (Tasks) s definovanou prioritou, které jsou spravovány plánovačem.

\subsection{Souborový systém LittleFS}
Pro ukládání konfigurace a offline dat (cache) je využit souborový systém \textbf{LittleFS}. 
Byl zvolen pro svou odolnost vůči výpadkům napájení (power-loss resilience), což je u bateriového zařízení kritické. 
Díky mechanismu \textit{Copy-on-Write} nedochází k poškození souborového systému ani při náhlém vypnutí.

\section{Návrh hardware - Výběr komponent}
\label{sec:hw_navrh}

Návrh hardwarové části systému vychází z požadavku na vytvoření nezávislého zařízení schopného provozu na baterii. 

\begin{figure}[H]
    \centering
    \includegraphics[width=0.7\linewidth]{image/T-Call-A7670-3936760664.jpg} 
    \caption{LilyGO T-Call V1.5 - integrované řešení ESP32 + LTE modem + použité piny}
    \label{fig:LilyGO_T-Call}
\end{figure}

\clearpage
\subsection{LilyGO T-Call v1.5}
Základem celého zařízení je vývojová deska \textbf{LilyGO T-Call} (verze V1.5), která v sobě integruje výkonný mikrokontrolér ESP32 a komunikační modem. Tato volba byla učiněna kvůli následujících kritérií:
\begin{itemize}
    \item \textbf{Integrace:} Spojení MCU a modemu na jedné desce eliminuje nutnost složitého propojování (bylo možné použít oddělene desku ESP32 a modem SIM800L, avšak zde by vznikly problémy s napájením. Modem totiž ve špičkách odebírá až 2A.)
    \item \textbf{Konektivita:} Použitý modem SIMCOM A7670 podporuje moderní sítě LTE (4G), což zajišťuje lepší pokrytí a nižší latenci než zastaralé 2G moduly (např. SIM800L).
\end{itemize}

V rané fázi projektu bylo zkoušeno používat desku LilyGO SIM800L (s podporou pouze 2G sítí), avšak kvůli vzniklému zkraty (typická to chyba) a nemožnosti opravy desky ručně (bylo by potřeba mikropájení), přešli jsme na desku T-call v1.5 s modemem A7670 (podpora LTE sítí).

\subsection{ Multi-GNSS L76K modul}

Pro získávání polohy byl zvolen externí modul \textbf{Multi-GNSS L76K} (hlavně kvůli dostupnosti).
Deska LilyGo T-call sice disponuje možností zisku GPS přes vestavěný modem, avšak externí modul nabízí lepší citlivost a
rychlost zisku dat (dle testů až 4x rychleji), také to, že takový modul byl ve vlastnitcví již před zahájením vývoje projektu, a 
tudíž nebylo nutné jej dokupovat prosadilo jeho použití. 

\begin{figure}[h]
	\centering
	\begin{subfigure}{0.48\linewidth}
		\centering
		\includegraphics[width=\linewidth]{image/multignsslk76.jpg}
		\caption{Externí modul Multi-GNSS L76K}
	\end{subfigure}
	\hfill
	\begin{subfigure}{0.48\linewidth}
		\centering
		\includegraphics[width=\linewidth]{../docs_hw/schemas/NPN_control.png}
		\caption{Ovladání GND}
	\end{subfigure}
	\caption{Modul GPS a zapojení pro řízení napájení}
	\label{fig:gps_module}
\end{figure}

Kvůli lepším možnostem ovladání spotřeby je modul GPS připojen přes tranzistorový spínač (NPN), který umožňuje mikrokontroléru zcela odpojit napájení GPS modulu, když není potřebný v zapnutém stavu.
Aplikován je princip řízeného GND, sice bylo rozvažováno nad stabilnější možností ovládat VCC (takto se může v krajních případech dostat do modulu napětí), 
avšak z důvodu absence vhodného PNP tranzistoru bylo zvoleno toto řešení.

\subsection{Baterie, TP4056 + MT3608}

Napájení zajišťuje Li-Ion článek typu 18650 (kapacita 3200 mAh zvolena kvůli ceně), který je dobíjen pomocí modulu s čipem \textbf{TP4056}. Pro zvýšení napětí z baterie (3.7V) na úroveň potřebnou pro stabilní provoz desky a GPS modulu(5V) je použit DC-DC step-up měnič \textbf{MT3608}.

\begin{figure}[H]
    \centering
    \begin{subfigure}{0.48\linewidth}
        \centering
        \includegraphics[width=\linewidth]{../docs_hw/screenshots/battery_carige.jpg}
        \caption{TP4056 + MT3608 v pouzdře}
        \label{fig:battery_charging_module_a}
    \end{subfigure}
    \hfill
    \begin{subfigure}{0.48\linewidth}
        \centering
        \includegraphics[width=\linewidth]{../docs_hw/screenshots/Power_desk.jpg}
        \caption{Osazená deska - Power Latch modul}
        \label{fig:battery_charging_module_b}
    \end{subfigure}
    \caption{Baterie a nabíjecí modul / power latch modul - osazená deska}
    \label{fig:battery_charging_module}
\end{figure}


\subsection{Řízení napájení - Power Latch modul}
V případě zařízení jakým je tracker, nestačí pouhé "tvrdé" odpojení od baterie vypínačem. Systém potřebuje čas na bezpečné ukončení procesů (uzavření souborů, odhlášení ze sítě).
K tomuto účelu slouží obvod \textbf{Power Latch} (samodržný obvod). Jeho primárním cílem je umožnit mikrokontroléru převzít kontrolu nad vlastním napájením.
Navržen byl dle rady a schématu od známeho v oboru Elektroniky. deska byla navržena v EasyEDA a vyrobena přes službu JLCPCB. Pro úsporu peněz byla osazena součástkami ručně (cena osazení cca. 4x převyšovala cenu součástek). Proto byly součástky vybrany dle požadavků na možnost ručního osazení (např. místo SMD vybrán mosfet v pouzdře TO-220 etc.).

Viz. implementace ve firmware, sekce \ref{subsec:napajeni_hw}.

\section*{Výsledná sestava - schéma zapojení - piny}

GPIO piny byly přiřazeny dle dostupnosti na desce, například pro GPS byly použity piny UART1 linky (GPIO16 - RX, GPIO17 - TX)
Pro vstup tlačítka byl použit pin GPIO32, který umožňuje vstupu i z režimu Deep Sleep (Ext0 Wakeup).
etc.

\begin{figure}[H]
    \centering
    \includegraphics[width=1\linewidth]{../docs_hw/schemas/schema_GENERAL_vyrez.jpg} 
    \caption{Blokové schéma zapojení hardwarového trackeru [easyEDA]}
    \label{fig:hw_schema_general}
\end{figure}

\begin{figure}[H]
    \centering
		\includegraphics[width=0.65\linewidth]{../docs_hw/schemas/schema_POWER_modu_cuttedl.jpg}
		\caption{Schéma power-modul [easyEDA]}
        \label{fig:hw_schema_power}
\end{figure}

\begin{figure}[h]
    \centering
    \includegraphics[width=0.8\linewidth]{../docs_hw/screenshots/foto_progresACTUAL.jpg} 
    \caption{"vnitřnosti" zařízení}
    \label{fig:class_dia}
\end{figure}

\section*{Seznam použitých součástek}
\begin{itemize}
  \setlength\itemsep{0.2em}
  \setlength\parskip{0pt}
  \setlength\parsep{0pt}
  \item LilyGO T-Call V1.5 (ESP32 + LTE modem A7670)
  \item Externí GPS modul Multi-GNSS L76K (alternativně NEO-6M)
  \item LTE mikroSIM karta, datový tarif (díky minimálnímu toku dat bylo za celé období testování využito do 1MB dat)
  \item GPS anténa (u.FL/SMA dle modulu)
  \item LTE anténa (u.FL/SMA dle desky)
  
  \item Li-Ion 18650 článek
  \item Nabíjecí modul TP4056 + Step-up měnič MT3608

  \item SEMTECH BC337-25 bipolární NPN tranzistor
  \item INFINEON IRF4905PBF P-Channel MOSFET
  \item Rezistory (1 k$\Omega$, 330 $\Omega$, 100 k$\Omega$)
  \item KLS 7-P8.0x8.0-0 non lock tlačítko do DPS, 2 póly, ON-(ON) 
  \item LED dioda 3mm zelená
\end{itemize}

\section*{Návrh \uv{krabice} zařízení - model}
V počátcích vývoje byl vytvořen návrh možného obalu pro komponenty zařízení, avšak s tím jak byla přehodnocována struktura zapojení a vyběr použitých součástek, tak model nestíhal držet krok. 
Při testovacím tisku a složení prototypu modelu také bylo zjištěno, že "Cable managment problem" si vyžaduje mnohem více pozornosti než bylo předvídáno. Model sice byl navržen kompaktně, ale v určitých případech až přespříliš, kdy nebylo možné nějakým postupem součásti v krabičce uzavřít bez nasilných úprav. 

\begin{figure}[h]
    \centering
    \includegraphics[width=0.55\linewidth]{../docs_hw/screenshots/box_prototyp.png} 
    \caption{prototyp krabičky - průhled}
    \label{fig:box_prototyp}
\end{figure}

\clearpage

\section{Implementace firmware}
\label{sec:fw_implementace}

Software pro mikrokontrolér byl vyvíjen v jazyce C++ s využitím frameworku Arduino. Jako vývojové prostředí bylo zvoleno \textbf{Visual Studio Code} s rozšířením \textbf{PlatformIO} kvůli největší známosti.
\subsection{Architektura a použité knihovny}
Projekt je koncipován modulárně, což usnadňuje orientaci v kódu a případné budoucí rozšiřování. 
Zdrojový kód je rozdělen do samostatných jednotek (hlavičkové a zdrojové soubory) podle funkčních celků:

\begin{itemize}
  \setlength\itemsep{0.2em}
  \setlength\parskip{0pt}
  \setlength\parsep{0pt}
    \item \texttt{main.cpp}: Hlavní smyčka programu, řízení stavového automatu.
    \item \texttt{gps\_control}: Obsluha GPS modulu, parsování NMEA zpráv, řízení napájení GPS.
    \item \texttt{modem\_control}: Komunikace s LTE modemem přes AT příkazy, správa GPRS připojení a HTTP požadavků.
    \item \texttt{file\_system}: Abstrakce nad souborovým systémem LittleFS, ukládání a načítání konfigurace a offline dat.
    \item \texttt{power\_management}: Řízení spotřeby, ovládání Power Latch obvodu, přechod do Deep Sleep.
    \item \texttt{ota\_mode}: Implementace servisního režimu, webového serveru a OTA aktualizací.
\end{itemize}

Mezi klíčové knihovny třetích stran, na kterých je firmware postaven, patří:
\begin{itemize}
  \setlength\itemsep{0.2em}
  \setlength\parskip{0pt}
  \setlength\parsep{0pt}
    \item \textbf{TinyGSM:} Univerzální knihovna pro komunikaci s GSM/LTE modemy. Pro tento projekt byl použit profil \texttt{TINY\_GSM\_MODEM\_A7670}.
    \item \textbf{TinyGPS++:} Efektivní parser dat z GPS modulu.
    \item \textbf{ArduinoJson:} Knihovna pro serializaci a deserializaci JSON objektů, používaná pro komunikaci s API serveru.
    \item \textbf{DeepsSleep.h:} Umožnuje použití "hlubokého spánku" v ESP32
    \item \textbf{Wifi.h:} knihovna pro připojení k WiFi síti, použitá v OTA režimu
    \item \textbf{FreeRTOS , LittleFS:} již zmíneny v sekci \ref{sec:hw_teorie}
\end{itemize}

\clearpage

\subsection{Pracovní cyklus}

Kvůli potřebě minimalizovat spotřebu baterie používá firmware funkcionalitu ESP32 : \textbf{DeepSleep} (hluboký spánek). viz \ref{sec:deep_sleep}.
Protože je použita tato funkcionalita, veškeré činosti zařízení se odehrávají v \textbf{\texttt{setup()}}.
Funkce \textbf{\texttt{loop()}}, obyčejně užita pro hlavní program, je nevyužita.
Veškerý kód je proto umístěn ve funkci \textbf{\texttt{setup()}}.

\noindent Standardní pracovní cyklus probíhá v následujících krocích:

\begin{enumerate}
  \setlength\itemsep{0.2em}
  \setlength\parskip{0pt}
  \setlength\parsep{0pt}
    \item \textbf{Probuzení a inicializace:} Po přivedení napájení (tlačítkem nebo časovačem) se provede inicializace periferií a načtení konfigurace ze souborového systému.
    \item \textbf{Získání polohy (GPS Fix):} Zapne se GPS modul. Mikrokontrolér čeká na platná data o poloze. Pokud není fix získán do stanoveného limitu (timeout), pokus se ukončí.
    \item \textbf{Uložení dat:} Získaná data (nebo informace o chybě) jsou uložena do interní paměti (LittleFS).
    \item \textbf{Odeslání dat (Upload):} Aktivuje se modem. Zařízení se připojí k mobilní síti a pokusí se odeslat dávku uložených záznamů na server.
    \item \textbf{Synchronizace:} Server potvrdí přijetí dat a případně pošle novou konfiguraci (např. změnu intervalu sledování).
    \item \textbf{Uspání (Deep Sleep):} Po dokončení všech úloh se zařízení odpojí od sítě, vypne periferie a přejde do režimu hlubokého spánku na dobu definovanou v konfiguraci.
\end{enumerate}

\clearpage

\begin{figure}[h]
    \centering
    \includegraphics[width=1\linewidth]{../docs_hw/schemas/life_cycle_sequence_dia.png} 
    \caption{sekvenční diagram životního cyklu}
    \label{fig:life_cycle_sequence_dia}
\end{figure}

\subsection{Paralelní zpracování a obsluha přerušení (FreeRTOS)}
Vzhledem k tomu, že ESP32 disponuje dvěma jádry používáme na něm operační systém FreeRTOS, který umožňuje efektivně rozdělit úlohy. Zatímco hlavní smyčka \texttt{loop()} řeší sekvenční logiku (GPS -> Modem -> Sleep), kritické události, jako je stisk tlačítka pro vypnutí, jsou řešeny asynchronně pomocí přerušení (ISR) a samostatné úlohy (Task).

Následující ukázka z \texttt{power\_management.cpp} ukazuje inicializaci této logiky. Funkce \texttt{xTaskCreatePinnedToCore} vytvoří novou úlohu \texttt{ShutdownTask}, která běží na pozadí a čeká na signál z přerušení tlačítka.

\begin{figure}[h]
    \centering
    \includegraphics[width=1\linewidth]{../docs_hw/screenshots/interupt_freertos.png} 
    \caption{funkce \texttt{power\_init()}}
    \label{fig:power_init}
\end{figure}

\subsection{Řízení napájení a Power Latch modul}
\label{subsec:napajeni_hw}
Princip celého systému napájení je následující:
\begin{enumerate}
    \item Uživatel stiskne tlačítko, čímž přivede napětí do systému.
    \item ESP32 se nastartuje a okamžitě nastaví pin \texttt{PIN\_EN} na úroveň HIGH. Tím "přemostí" tlačítko a drží se pod napětím i po jeho uvolnění.
    \item Když chce zařízení přejít do vypnutého stavu (Deep Sleep nebo úplné vypnutí), provede potřebné úkony a následně nastaví \texttt{PIN\_EN} na LOW, čímž se samo odpojí.
\end{enumerate}

Následující ukázka kódu demonstruje funkci \texttt{graceful\_shutdown()}, která zajišťuje bezpečné vypnutí.

\begin{figure}[h]
    \centering
    \includegraphics[width=0.8\linewidth]{../docs_hw/screenshots/gracefull_shutdown.png} 
    \caption{funkce \texttt{graceful\_shutdown()}}
    \label{fig:graceful_shutdown}
\end{figure}

\subsubsection{Režim hlubokého spánku (Deep Sleep)}
\label{sec:deep_sleep}
Kromě úplného vypnutí (Power Off) využívá zařízení také režim hlubokého spánku. Tento režim používáme pro cyklické probouzení trackeru (jinak by bylo nutné pokaždé stisknout tlačítko pro zapnutí, což je očividně nepoužitelné).
V režimu Deep Sleep je vypnuto CPU, RAM i většina periferií. Napájen zůstává pouze RTC (Real Time Clock) řadič a malá část paměti (RTC Slow Memory). Spotřeba čipu v tomto stavu klesá na jednotky mikroampérů.

Probuzení z tohoto režimu může nastat dvěma způsoby:
\begin{enumerate}
    \item \textbf{Časovač (Timer Wakeup):} Po uplynutí nastaveného intervalu (např. 10 minut).
    \item \textbf{Externí signál (Ext0 Wakeup):} Stiskem tlačítka (změna logické úrovně na pinu GPIO32).
\end{enumerate}

Následující funkce \texttt{enter\_deep\_sleep} ukazuje konfiguraci těchto budících zdrojů před uspáním procesoru.

\begin{figure}[H]
    \centering
    \includegraphics[width=0.7\linewidth]{../docs_hw/screenshots/deep_sleep.png} 
    \caption{funkce \texttt{enter\_deep\_sleep()}}
    \label{fig:enter_deep_sleep}
\end{figure}

\clearpage

\subsection{Zpracování GPS dat a parsování NMEA}
Komunikace s GPS modulem L76K probíhá přes hardwarovou sériovou linku (UART1). Modul v pravidelných intervalech (1 Hz dle dokumentace) odesílá textová data ve formátu NMEA. Pro zpracování těchto dat to rozumné podoby jsme využili knihovnu \textbf{TinyGPS++}.

Klíčovou částí implementace je funkce \texttt{gps\_get\_fix}, která v cyklu čte znaky ze sériového portu a předává je parseru pomocí metody \texttt{gps.encode()}. Jakmile parser detekuje platnou větu a aktualizuje souřadnice, zkontroluje se, zda jsou splněny podmínky pro platný fix (validní poloha, čas a minimální počet satelitů).

\begin{figure}[H]
    \centering
    \includegraphics[width=0.8\linewidth]{../docs_hw/screenshots/get_gps_fix.png} 
    \caption{funkce \texttt{gps\_get\_fix()}}
    \label{fig:gps_get_fix}
\end{figure}

\subsection{Správa datového úložiště (LittleFS)}
Jedním z klíčových požadavků byla schopnost pracovat i v oblastech bez signálu mobilní sítě nebo v případech výpadku připojení nebo selhání modemu. Firmware proto implementuje cachování dat.
Naměřené polohy nejsou odesílány okamžitě, ale jsou nejprve serializovány a uloženy do souboru \texttt{/gps\_cache.log} v paměti flash (LittleFS). Pro tento učel musel být systém inicializován pomocí vytvoření složky \textbf{data/} ve složce firmwaru.
Při každém úspěšném připojení k internetu se pak zařízení pokusí odeslat dávku nejstarších záznamů (pravidlo : FIFO - First In, First Out). Teprve po potvrzení serverem (success: true) jsou zaznamy z cache smazány.

\noindent   Následuje přiklad funkce  texttt{fs\_init()} jež zajištuje nabootovaní souborového systému.

\begin{figure}[H]
    \centering
    \includegraphics[width=1\linewidth]{../docs_hw/screenshots/fs_init.PNG} 
    \caption{funkce \texttt{fs\_init()}}
    \label{fig:fs_init}
\end{figure}


\section{Komunikace a data}
\label{sec:hw_komunikace}

\subsection{Komunikační protokol a zabezpečení}
Komunikace mezi zařízením a serverem probíhá přes protokol HTTPS (modem SIMCOM A7670 umožnuje použití šifrování narozdíl třeba od SIM800L).

Pro navázání zabezpečeného spojení využívá modem knihovnu SSL/TLS. 
Následující ukázka kódu z funkce \texttt{modem\_send\_post\_request} ukazuje inicializaci HTTPS relace a nastavení cílové URL.

\begin{figure}[h]
    \centering
    \includegraphics[width=0.7\linewidth]{../docs_hw/screenshots/https_session.png} 
    \caption{funkce \texttt{modem\_send\_post\_request()}}
    \label{fig:https_post_request}
\end{figure}

\subsection{Struktura dat a handshake}
Komunikace se serverem probíhá prostřednictvím REST API rozhraní. Zařízení odesílá data ve formátu JSON (JavaScript Object Notation).

\noindent Detailní specifikace jednotlivých endpointů, struktura přenášených zpráv a návratové kódy jsou podrobně popsány v kapitole věnované serverové části (viz sekce \ref{sec:server_api}).

Pro operace a práci s odesílanými a přijímanými JSON daty je využita knihovna \textbf{ArduinoJson}. Následující ukázka kódu z funkce \texttt{modem\_perform\_handshake} demonstruje sestavení JSON objektu pro úvodní "pozdrav" (handshake) a zpracování odpovědi.

\begin{figure}[H]
    \centering
    \includegraphics[width=1\linewidth]{../docs_hw/screenshots/perform_handshake.png} 
    \caption{funkce \texttt{modem\_perform\_handshake()}}
    \label{fig:handshake_json}
\end{figure}

\clearpage

\subsection{Práce s Modemem a AT příkazy}
Komunikace s LTE modemem SIMCOM A7670 probíhá prostřednictvím sériové linky (UART) pomocí sady standardizovaných AT příkazů. 
Firmware využívá knihovnu \textbf{TinyGSM}, která umožňuje využití vysokoúrovňových metod (např. \texttt{gprsConnect}, \texttt{https\_post}) místo čistých AT příkazů.

Přesto je v některých případech nutné používat AT příkazy přímo, například při inicializaci modemu. 
Následující ukázka z funkce \texttt{modem\_initialize} ukazuje sekvenci AT příkazů pro ověření dostupnosti modemu a jeho zapnutí.

\begin{figure}[H]
    \centering
    \includegraphics[width=1\linewidth]{../docs_hw/screenshots/modem_init_AT.png} 
    \caption{funkce \texttt{modem\_initialize()}}
    \label{fig:modem_initialize}
\end{figure}

Knihovna TinyGSM následně volání funkcí překládá na konkrétní AT příkazy. Například volání \texttt{g\_modem.getSignalQuality()} odešle příkaz \texttt{AT+CSQ} a vyparsuje odpověď ve formátu \texttt{+CSQ: <rssi>,<ber>}.



\subsection{OTA režim a konfigurační rozhraní}
\label{sec:hw_ota_mode}

Pro prvotní registraci, přenastavení konfigurace nebo nahrání nového firmwaru (v BIN formě), zařízení disponuje režimem "OTA" (Over-The-Air). Tento režim se aktivuje dlouhým stiskem ovládacího tlačítka při startu zařízení (cca 2 sekundy).

Detekce tohoto stavu probíhá v nejranější fázi funkce \texttt{setup()}, ještě před inicializací ostatních periferií. Pokud je detekován dlouhý stisk, zařízení nespustí standardní pracovní cyklus, ale přejde do smyčky OTA režimu.

V OTA režimu se ESP32 přepne do role Wi-Fi přístupového bodu (Access Pointu) s názvem \texttt{lotrTrackerOTA\_<DeviceID> (získano z MAC adresy)} a spustí webový server na adrese \texttt{192.168.4.1}. Uživatel se může připojit na tuto wifi síť a přes webové rozhraní provádět změny.

\begin{figure}[H]
    \centering
    \includegraphics[width=0.5\linewidth]{../docs_hw/screenshots/apnwifi.png} 
    \caption{Ukázka Wi-Fi}
    \label{fig:wifi_apn_phone}
\end{figure}

Webové rozhraní nabízí následující funkce:
\begin{itemize}
  \setlength\itemsep{0.2em}
  \setlength\parskip{0pt}
  \setlength\parsep{0pt}
    \item \textbf{Registration:} Spárování nového zařízení s uživatelským účtem.
    \item \textbf{Settings:} Konfigurace APN, adresy serveru a portu.
    \item \textbf{Test GPRS \& Test Serverové konektivity:} Okamžité ověření konektivity modemu a dostupnost serveru.
    \item \textbf{Firmware Update:} Nahrání nové binární verze firmware přímo z prohlížeče. (s následným restartem zařízení)
    \item \textbf{Clear Cache:} Pokud je potřeba, umožňuje vymazat uložená data v paměti.(např. pokud jsou data poškozena či ve špatném formátu a nelze je odeslat). používá funkce LittleFS.
\end{itemize}

\begin{figure}
    \centering
    \includegraphics[width=0.9\linewidth]{../docs_hw/screenshots/wifi_apn.png} 
    \caption{konfigurace Wifi APN pro OTA režim}
    \label{fig:wifi_apn}
\end{figure}


\subsection{Uložení konfigurace zařízení}
Konfigurace, ať již nastavena v \texttt{/settings} (OTA režim) nebo samotné nastavení posílané Serverem, je trvale uložena v paměti pomocí knihovny \texttt{Preferences}. Mezi klíčové parametry patří:
\begin{itemize}
    \setlength\itemsep{0.2em}
  \setlength\parskip{0pt}
  \setlength\parsep{0pt}
    \item \texttt{apn}, \texttt{gprsUser}, \texttt{gprsPass}: Nastavení mobilní sítě.
    \item \texttt{server}, \texttt{port}: Cílová adresa backendu.
    \item \texttt{sleepTimeSeconds}: Interval mezi probuzeními.
    \item \texttt{minSatellitesForFix}: Minimální počet satelitů pro validní fix.
    \item \texttt{etc...}
\end{itemize}


\section{Obrázky}

%\begin{figure}[h]
%    \centering
%    \includegraphics[width=0.8\linewidth]{../docs_hw/screenshots/OTA_log_1.png}
%    \caption{log z OTA režimu}
%    \label{fig:ota_log}
%\end{figure}

\begin{figure}[h]
  \centering
  \begin{subfigure}{0.48\linewidth}
    \centering
    \includegraphics[width=\linewidth]{../docs_hw/screenshots/OTA1.png}
    \caption{registrace}
    \label{fig:ota_Registration}
  \end{subfigure}
  \hfill
  \begin{subfigure}{0.48\linewidth}
    \centering
    \includegraphics[width=\linewidth]{../docs_hw/screenshots/OTA3.png.jpg}
    \caption{OTA režim - webové rozhraní}
  \end{subfigure}
  \caption{OTA režim - webové rozhraní}
  \label{fig:ota_Settings}
\end{figure}


%\begin{figure}[h]
%   \centering
%   \includegraphics[width=1\linewidth]{../docs_hw/schemas/process_flow.png} 
%   \caption{Stavový diagram trackeru}
%   \label{fig:process_flow_tracker}
%\end{figure}

\cleardoublepage

\chapter{Serverová část}

\section{Úvod a koncepce systému}
\label{sec:server_uvod}
Serverová část systému LOTR představuje centrální bod celého systému. Zajišťuje komunikaci s hardwarovými jednotkami, trvalé ukládání telemetrických dat, správu uživatelských účtů a poskytování uživatelského rozhraní pro vizualizaci polohy a stavu zařízení.

\subsection{Role serveru v systému LOTR}
Hlavním úkolem serveru je agregace dat z jednotlivých GPS trackerů. Hardwarové jednotky odesílají data (poloha, stav baterie, síla signálu) prostřednictvím mobilní sítě na definované API endpointy. Server tato data validuje, zpracovává a ukládá do relační databáze.
Druhým klíčovým úkolem je obsluha klientských požadavků. Uživatelé přistupují k systému prostřednictvím webového prohlížeče. Server zajišťuje autentizaci uživatelů a generuje dynamické HTML stránky zobrazující mapové podklady s aktuální polohou sledovaných objektů.

\subsection{Monolitická architektura a MVC vzor}
Aplikace je navržena jako monolit, což znamená, že backendová logika i prezentační vrstva (frontend) jsou součástí jednoho projektu a běží v rámci jednoho procesu. Pro organizaci kódu byl zvolen architektonický vzor \textbf{MVC (Model-View-Controller)}, který odděluje data, logiku a zobrazení:

\begin{itemize}
    \item \textbf{Model (Model):} Definuje strukturu dat a logiku pro přístup k databázi. V našem případě je tato vrstva realizována pomocí ORM knihovny Sequelize (viz sekce \ref{subsec:server_db}). Modely odpovídají tabulkám v databázi (např. \texttt{User}, \texttt{Device}, \texttt{Telemetry}).
    \item \textbf{View (Pohled):} Stará se o prezentaci dat uživateli. Využíváme šablonovací systém EJS, který umožňuje vkládat data z backendu přímo do HTML struktury.
    \item \textbf{Controller (Řadič):} Přijímá požadavky od uživatele (nebo API), zpracovává je s využitím Modelů a rozhoduje, jaký Pohled se má zobrazit, nebo jaká data se mají vrátit (v případě JSON API).
\end{itemize}

Tento přístup usnadňuje údržbu kódu a umožňuje paralelní vývoj jednotlivých částí aplikace.

\subsection{Relační databáze a ORM (MySQL + Sequelize)}
Pro ukládání strukturovaných dat s jasně definovanými vazbami jsou standardem relační databáze (RDBMS). Systém \textbf{MySQL} poskytuje transakční zpracování dat a dodržuje principy ACID (Atomicity, Consistency, Isolation, Durability), což je nezbytné pro zajištění integrity telemetrických dat.
Pro interakci s databází se v moderním vývoji často využívá technika \textbf{ORM} (Object-Relational Mapping). Knihovna (zde Sequelize) mapuje databázové tabulky na třídy (objekty) v programovacím jazyce. Vývojář tak nemusí psát surové SQL dotazy, ale manipuluje s daty pomocí metod objektů (např. \texttt{User.findAll()}).

\subsection{Principy autentizace a OAuth 2.0}
Zabezpečení přístupu k API a webovému rozhraní vyžaduje robustní autentizační mechanismus. Kromě klasického ověření jménem a heslem se stále častěji využívá standard \textbf{OAuth 2.0}.
Tento protokol umožňuje uživateli udělit aplikaci přístup ke svým údajům na jiné službě (např. Google, GitHub) bez nutnosti sdílet své heslo. Aplikace získá pouze tzv. \textit{Access Token}, který slouží k ověření identity.
Implementováno pomocí knihovny \texttt{passport.js}.

\subsection{REST API architektura}
\textbf{REST} (Representational State Transfer) je architektonický styl pro návrh síťových aplikací. REST API definuje sadu pravidel pro komunikaci mezi klientem a serverem:
\begin{itemize}
    \item \textbf{Bezstavovost (Stateless):} Server neuchovává stav klienta mezi požadavky. Každý požadavek musí obsahovat všechny potřebné informace (např. autorizační token).
    \item \textbf{Jednotné rozhraní:} Zdroje (data) jsou identifikovány pomocí URL a manipuluje se s nimi pomocí standardních HTTP metod (GET pro čtení, POST pro vytvoření, PUT pro úpravu, DELETE pro smazání).
\end{itemize}

\section{Návrh a implementace backendu}
\label{sec:server_backend}

\subsection{Databázová vrstva (MySQL a Sequelize)}
\label{subsec:server_db}
Databázová vrstva je postavena na relační databázi MySQL. Pro komunikaci s databází je využita knihovna \texttt{mysql2} ve spojení s ORM frameworkem Sequelize.

\subsubsection{ORM Sequelize a definice modelů}
Sequelize abstrahuje SQL dotazy do JavaScriptových objektů. V projektu jsou definovány následující klíčové modely:
\begin{itemize}
    \item \texttt{User}: Ukládá informace o uživatelích (jméno, email, hash hesla, OAuth ID).
    \item \texttt{Device}: Reprezentuje hardwarové jednotky (tracker). Obsahuje unikátní identifikátor, název a vazbu na vlastníka (\texttt{User}).
    \item \texttt{Location}: Uchovává historická data o poloze (zeměpisná šířka, délka, čas, rychlost, stav baterie).
    \item \texttt{Alert}: Záznamy o bezpečnostních událostech (např. opuštění geofence zóny).
\end{itemize}
Vztahy mezi modely jsou definovány jako 1:N (jeden uživatel má více zařízení, jedno zařízení má více záznamů polohy).

\begin{figure}[H]
    \centering
    \includegraphics[width=1\textwidth]{../docs_server/db_diagrams/DB_scheme.png} 
    \caption{ER diagram databázových modelů}
    \label{fig:db_schema}
\end{figure}

\subsubsection{API pro HW tracker (ESP32)}
HW klient se autentizuje pouze pomocí \texttt{device\_id} registrovaného u uživatele. Nevyužívá session/cookie, neodesílá uživatelská hesla.

\begin{table}[H]
\centering
\begin{tabular}{|l|l|p{7cm}|}
\hline
Endpoint & Metoda & Popis \\ \hline
\verb|/api/devices/register| & POST & Registrace HW zařízení k účtu. Payload: \verb|client_type="HW"|, \verb|device_id|, \verb|username|, \verb|password|, volitelné \verb|name|. \\ \hline
\verb|/api/devices/handshake| & POST & Periodická synchronizace stavu a konfigurace. Payload: \verb|device_id|, \verb|client_type="HW"|, \verb|power_status|. Odezva obsahuje \verb|registered|, \verb|config|, \verb|power_instruction|. \\ \hline
\verb|/api/devices/input| & POST & Příjem telemetrie. Payload: jeden objekt nebo pole objektů se souřadnicemi, časem, rychlostí, \verb|power_status|. \\ \hline
\end{tabular}
\caption{HW API endpointy}
\label{tab:api_hw}
\end{table}

\begin{enumerate}
    \item Každý záznam musí obsahovat \texttt{device}, \texttt{latitude}, \texttt{longitude}, \texttt{timestamp}, může obsahovat \texttt{power\_status}.
    \item Server uloží lokace, aktualizuje \texttt{last\_seen}; pokud \texttt{power\_status} potvrzuje instrukci, vynuluje \texttt{power\_instruction}.
    \item Odezva \texttt{"success": true}; až poté klient smaže data z bufferu.
\end{enumerate}

Chybové stavy (HW):
\begin{itemize}
  \setlength\itemsep{0.2em}
  \setlength\parskip{0pt}
  \setlength\parsep{0pt}
    \item \texttt{404 / registered=false}: zařízení není známo; má přejít do konfiguračního režimu.
    \item \texttt{409}: \texttt{device\_id} patří jinému účtu; zařízení musí zastavit akce.
    \item \texttt{500+}: klient retry s backoff; data v bufferu se nemažou.
\end{itemize}

\subsubsection{API pro APK klient (Android)}
APK klient používá uživatelské přihlášení (session cookie) a odděleně identifikuje zařízení pomocí \texttt{device\_id = installationId}. Telemetrie i handshake sdílí endpointy s HW, ale autentizační vrstva je odlišná.

\begin{table}[H]
\centering
\begin{tabular}{|l|l|p{7cm}|}
\hline
Endpoint & Metoda & Popis \\ \hline
    \texttt{/api/apk/login} & POST & Přihlášení uživatele; vyžaduje ověřený účet (\texttt{user.is\_verified=true}); vydá HTTP-only cookie \texttt{connect.sid}. \\ \hline
	\texttt{/api/apk/logout} & POST & Zrušení session; odhlášení APK klienta. \\ \hline
	\texttt{/api/devices/register} & POST & Registrace zařízení typu APK. Payload: \texttt{client\_type="APK"}, \texttt{device\_id} (UUID), \texttt{name}; vyžaduje platnou session. \\ \hline
	\texttt{/api/devices/handshake} & POST & Synchronizace konfigurace a power instrukcí. Payload: \texttt{device\_id}, \texttt{client\_type="APK"}, \texttt{power\_status}, \texttt{reason}. \\ \hline
	\texttt{/api/devices/input} & POST & Odeslání telemetrie z APK (batch array). Stejný formát jako HW, navíc může obsahovat \texttt{accuracy}. \\ \hline
	\texttt{/api/devices/coordinates} & GET & Vrací poslední známé polohy všech zařízení přihlášeného uživatele (používá se pro mapu ve webu/APK). \\ \hline
\end{tabular}
\caption{APK API endpointy}
\label{tab:api_apk}
\end{table}

APK Handshake (stejný endpoint, odlišný kontext): APK spouští handshake při startu služby, periodicky (např. 15 min) a po odeslání dávky. Server vrací \texttt{registered}, \texttt{config} a \texttt{power\_instruction}; APK musí respektovat \texttt{TURN\_OFF} zastavením služby.

APK Input: Telemetrie se odesílá dávkově z lokální SQLite (store-and-forward). Každý záznam obsahuje \texttt{device}, \texttt{latitude}, \texttt{longitude}, \texttt{timestamp}, \texttt{power\_status}, případně \texttt{speed} a \texttt{accuracy}. Server po \texttt{success:true} dovolí klientovi smazat dávku.

Autentizace a chyby (APK):
\begin{itemize}
      \setlength\itemsep{0.2em}
  \setlength\parskip{0pt}
  \setlength\parsep{0pt}
        \item \texttt{401/403}: neplatná nebo chybějící session; APK musí zneplatnit lokální session (logout broadcast).
        \item \texttt{404 registered=false}: zařízení není registrováno; klient zastaví službu a vyžádá novou registraci.
        \item \texttt{500+}: retry s exponential backoff; data v lokální DB zůstávají.
\end{itemize}

\subsubsection{Validace vstupních dat (Express Validator)}
Pro všechny vstupy (HW i APK) platí validace přes \texttt{express-validator}. Chybné payloady jsou odmítnuty kódem 400 ještě před zpracováním.

\begin{figure}[h]
    \centering
    \begin{subfigure}{0.48\textwidth}
        \centering
        \includegraphics[width=\linewidth]{../docs_server/screenshots/HW_server_respo.png}
        \caption{HW Handshake tok}
        \label{fig:hw_handshake_flow}
    \end{subfigure}
    \hfill
    \begin{subfigure}{0.48\textwidth}
        \centering
        \includegraphics[width=\linewidth]{../docs_server/screenshots/HW_APK_input.png}
        \caption{HW/APK Input tok}
        \label{fig:hw_apk_input_flow}
    \end{subfigure}
\end{figure}

\subsubsection{Dokumentace API (Swagger)}
Swagger (\texttt{swagger-jsdoc} + \texttt{swagger-ui-express}) generuje strojově čitelnou dokumentaci z komentářů controllerů a je dostupný na \texttt{/api-docs}.

\begin{figure}[h]
    \centering
    \includegraphics[width=1\textwidth]{../docs_server/screenshots/swagger_docs.png} 
    \caption{Ukázka Swagger dokumentace API}
    \label{fig:swagger_api}
\end{figure}

\clearpage

\subsection{Správa uživatelů a Autorizace}
\label{subsec:server_users}
Používáme uživatelské účty s možností registrace, přihlášení a správy jejich údajů. Pro bezpečnou autentizaci a autorizaci je implementován middleware \texttt{authorization.js} (popsáno výše).

\subsubsection{Autorizace a role (authorization.js)}
Middleware \texttt{authorization.js} vynucuje přihlášení, odděluje běžné uživatele, administrátory a blokuje nevhodné akce ROOT účtu, a zároveň autentizuje HW/APK zařízení na základě \texttt{device\_id}:
\begin{itemize}
    \setlength\itemsep{0.2em}
    \item \texttt{isAuthenticated}/\texttt{isApiAuthenticated}: vyžadují platnou session pro web i API.
    \item \texttt{isUser}/\texttt{isRoot}/\texttt{isNotRootApi}: směrují podle role, zakazují ROOT na běžných API a naopak.
    \item \texttt{authenticateDevice}: pro HW/APK čte \texttt{device\_id}, ověří vazbu na \texttt{User} a naplní \texttt{req.device}, \texttt{req.user}, \texttt{req.clientType}.
\end{itemize}

\begin{table}[H]
\centering
\begin{tabular}{|l|p{5cm}|p{5cm}|}
\hline
Role / middleware & Webové routy & API routy \\ \hline
	\texttt{isAuthenticated} & Zobrazí HTML nebo přesměruje na \texttt{/login}. & Vrací 401 JSON, neprovádí přesměrování. \\ \hline
	\texttt{isUser} & Povolen běžný uživatel, \texttt{root} přesměrován do administrace. & Kombinuje se s \texttt{isNotRootApi} pro blokaci \texttt{root} na uživatelských API. \\ \hline
	\texttt{isRoot} & Otevírá \texttt{/administration}. & Chrání \texttt{/api/admin/*}. \\ \hline
	\texttt{authenticateDevice} & N/A & Povinné pro \texttt{/api/devices/input|handshake}, váže \texttt{device\_id} k uživateli. \\ \hline
\end{tabular}
\caption{Rychlá orientace v rolích a middleware}
\label{tab:auth_roles}
\end{table}

\begin{figure}[h]
    \centering
    \includegraphics[width=0.75\textwidth]{../docs_server/screenshots/authorization_example.png} 
    \caption{příklad jedné z funkcí v authorization.js}
    \label{fig:auth_middleware}
\end{figure}

\subsubsection{Session a cookie politika}
Session jsou spravovány pomocí \texttt{express-session}. Klíč \texttt{SESSION\_SECRET} se načítá z prostředí, cookie je \texttt{httpOnly} s \texttt{maxAge} 6 hodin. Flag \texttt{secure} se zapíná, pokud je \texttt{NODE\_ENV=using\_ssl}, aby se cookie přenášela jen přes HTTPS. Webové routy používají \texttt{isAuthenticated} a případně přesměrování, API vrací JSON přes \texttt{isApiAuthenticated}.

\subsubsection{Správa identit (Passport.js)}
Knihovna \texttt{passport.js} zajišťuje flexibilní autentizaci. V systému jsou implementovány tři strategie:
\begin{itemize}
      \setlength\itemsep{0.2em}
  \setlength\parskip{0pt}
  \setlength\parsep{0pt}
    \item \textbf{Local Strategy:} Přihlášení pomocí emailu a hesla.
    \item \textbf{Google Strategy:} OAuth 2.0 přihlášení přes Google účet (\texttt{passport-google-oauth20}).
    \item \textbf{GitHub Strategy:} OAuth 2.0 přihlášení přes GitHub účet (\texttt{passport-github2}).
\end{itemize}
Po úspěšném přihlášení je uživatelská relace (session) uložena a identifikována pomocí cookie.

\begin{figure}[h]
    \centering
    \includegraphics[width=0.75\textwidth]{../docs_server/screenshots/no-local_strategies.png} 
    \caption{Passport.js autentizační strategie}
    \label{fig:passport_strategies}
\end{figure}
\subsubsection{Hashování hesel (Bcrypt.js)}
Hesla uživatelů nejsou nikdy ukládána v otevřené podobě. Při registraci je heslo prohnáno hashovací funkcí \texttt{bcrypt} se solí (salt). Při přihlášení se zadané heslo zahashuje a porovná s uloženým hashem. To chrání uživatele i v případě úniku databáze.


\subsubsection{ROOT účet} Z bezpečnostních i provozních důvodů systém startuje s hardcoded administrátorským účtem \texttt{ROOT}. Tento účet má vlástní administraci, která umožnuje přímí přístup do DB a manipulaci s jejími daty.

\begin{figure}[h]
    \centering
    \includegraphics[width=1\textwidth]{../docs_server/screenshots/root.png} 
    \caption{Administrátorský panel}
    \label{fig:admin_panel}
\end{figure}

\subsubsection{E-mailová komunikace (Nodemailer)}
Pro interakci s uživatelem mimo webové rozhraní slouží knihovna \texttt{nodemailer}. Využívá se v následujících scénářích:
\begin{itemize}
          \setlength\itemsep{0.2em}
  \setlength\parskip{0pt}
  \setlength\parsep{0pt}
    \item \textbf{Verifikace emailu:} Po registraci je odeslán unikátní kód pro ověření existence emailové schránky.
    \item \textbf{Reset hesla:} Odeslání odkazu pro obnovu zapomenutého hesla.
    \item \textbf{Bezpečnostní alerty:} Upozornění uživatele, pokud jeho zařízení opustí nastavenou bezpečnou zónu (Geofence).
\end{itemize}

\begin{figure}[h]
    \centering
    \begin{subfigure}{0.48\textwidth}
        \centering
        \includegraphics[width=\linewidth]{../docs_server/screenshots/nodemailer_trans.png}
        \caption{Použití Nodemailer.js}
        \label{fig:nodemailer_email}
    \end{subfigure}
    \hfill
    \begin{subfigure}{0.48\textwidth}
        \centering
        \includegraphics[width=\linewidth]{../docs_server/screenshots/warning_email.png}
        \caption{Upozornění na opuštění geofence zóny}
        \label{fig:geofence_email}
    \end{subfigure}
    \caption{E-mailové scénáře: transakční zprávy a bezpečnostní upozornění}
\end{figure}

\section{Prezentační vrstva (Frontend)}
\label{sec:server_frontend}
Frontendová část aplikace je navržena s důrazem na jednoduchost a rychlou odezvu. Kombinuje server-side rendering (SSR) pro základní strukturu stránky a klientský JavaScript pro dynamické aktualizace dat v reálném čase.

\subsection{Šablonovací systém EJS a struktura pohledů}
\label{subsec:server_ejs}
Pro generování HTML stránek na straně serveru je použit šablonovací systém **EJS (Embedded JavaScript)**. Ten umožňuje vkládat JavaScriptovou logiku přímo do HTML kódu.
Struktura pohledů je modularizována pomocí tzv. \textit{partials} (dílčích šablon), což zabraňuje duplicitě kódu. Typická stránka se skládá z:
\begin{itemize}
  \setlength\itemsep{0.2em}
  \setlength\parskip{0pt}
  \setlength\parsep{0pt}
    \item \texttt{\_head.ejs}: Meta tagy, importy CSS stylů a externích knihoven.
    \item \texttt{\_navbar.ejs}: Navigační lišta s odkazy a informacemi o přihlášeném uživateli.
    \item \textbf{Obsah stránky:} Unikátní obsah pro daný pohled (např. \texttt{index.ejs} pro mapu, \texttt{settings.ejs} pro nastavení).
    \item \texttt{\_footer.ejs}: Patička stránky a importy JavaScriptových souborů.
\end{itemize}
Data z backendu (např. seznam zařízení, chybové hlášky) jsou do šablon předávána při vykreslování v controlleru.

\subsection{Vizualizace dat a mapové podklady}
\label{subsec:server_maps}
Klíčovou funkcí frontendu je vizualizace polohy trackerů na mapě. Pro tento účel byla zvolena open-source knihovna **Leaflet.js**, která je lehká a flexibilní.
Jako zdroj mapových podkladů (dlaždic) slouží služba **OpenStreetMap**.

\subsubsection{Dynamická aktualizace polohy (AJAX)}
Aby uživatel viděl aktuální polohu zařízení bez nutnosti obnovovat celou stránku, využívá aplikace technologii **AJAX** (Asynchronous JavaScript and XML) prostřednictvím moderního Fetch API.

\begin{enumerate}
    \item Při načtení stránky se inicializuje mapa a vykreslí se poslední známé polohy.
    \item Klientský skript (\texttt{index.js}) spouští v pravidelných intervalech (nastaveno na 5 sekund) požadavek na API endpoint \texttt{/api/devices/coordinates}.
    \item Server vrátí aktuální data ve formátu JSON.
    \item JavaScript na klientovi porovná nová data s existujícími markery na mapě a aktualizuje jejich pozici, případně obsah informačních bublin (tooltipů).
\end{enumerate}

\begin{figure}[h]
    \centering
    \includegraphics[width=0.9\textwidth]{../docs_server/screenshots/manage.png}
    \caption{Webové rozhraní - Správa zařízení}
    \label{fig:web_manage}
\end{figure}

\begin{figure}[h]
    \centering
    \includegraphics[width=0.9\textwidth]{../docs_server/screenshots/settings.png}
    \caption{Webové rozhraní - Settings}
    \label{fig:web_settings}
\end{figure}

\section{Nasazení a provoz (Deployment)}
\label{sec:server_deployment}
Pro zajištění konzistentního běhového prostředí a snadného nasazení na různé servery (vývojový, produkční) je celý systém kontejnerizován pomocí technologie **Docker**. Kontejnerizace eliminuje problémy typu "u mě to funguje", protože aplikace si nese veškeré své závislosti (Node.js runtime, knihovny) s sebou.

\subsection{Docker a kontejnerizace}
\label{subsec:server_docker}
Obraz (image) serverové aplikace je definován v souboru \texttt{Dockerfile}. Jako základ je použit oficiální odlehčený obraz \texttt{node:24-slim}, který minimalizuje velikost výsledného kontejneru.
Proces sestavení obrazu zahrnuje:
\begin{enumerate}
  \setlength\itemsep{0.2em}
  \setlength\parskip{0pt}
  \setlength\parsep{0pt}
    \item Nastavení pracovního adresáře na \texttt{/app}.
    \item Kopírování definice závislostí (\texttt{package.json}) a jejich instalace pomocí \texttt{npm install}.
    \item Kopírování zdrojových kódů aplikace.
    \item Expozice portu 5000, na kterém server naslouchá.
    \item Definice spouštěcího příkazu \texttt{node server.js}.
\end{enumerate}

\begin{figure}[h]
    \centering
    \includegraphics[width=0.4\textwidth]{../docs_server/screenshots/dockerfile.png} 
    \caption{Ukázka Dockerfile pro serverovou aplikaci}
    \label{fig:dockerfile_example}
\end{figure}

\subsection{Orchestrace kontejnerů (Docker Compose)}
\label{subsec:server_compose}
Protože systém vyžaduje ke svému běhu nejen aplikační server, ale i databázi, využíváme nástroj Docker Compose pro definici a spouštění více kontejnerů současně. Konfigurace je uložena v souboru \texttt{docker-compose.yml}, který definuje dvě hlavní služby:

\begin{itemize}
    \item \textbf{app:} Samotná Node.js aplikace. Služba je nakonfigurována tak, aby čekala na plné spuštění databáze (\texttt{depends\_on}). Skrze proměnné prostředí (environment variables) jsou předány konfigurační údaje jako přístup k DB, tajné klíče pro sessions a nastavení CORS.
    \item \textbf{mysql:} Databázový server MySQL verze 8.0. Data jsou ukládána do trvalého svazku (volume) \texttt{mysql-data}, což zajišťuje, že data přežijí i restart nebo smazání kontejneru. Při prvním spuštění se automaticky provede inicializační skript \texttt{init-db.sql}.
\end{itemize}

Obě služby běží ve společné virtuální síti \texttt{gps-network}, což jim umožňuje vzájemnou komunikaci pomocí názvů služeb (hostname), aniž by musely být vystaveny do veřejného internetu (s výjimkou aplikačního portu 5000).


\begin{figure}[h]
    \centering
    \begin{subfigure}{0.48\textwidth}
        \centering
        \includegraphics[width=\linewidth]{../docs_server/screenshots/registration.png}
        \caption{Registrace}
        \label{fig:registration}
    \end{subfigure}
    \hfill
    \begin{subfigure}{0.48\textwidth}
        \centering
        \includegraphics[width=\linewidth]{../docs_server/screenshots/login.png}
        \caption{Přihlášení}
        \label{fig:login}
    \end{subfigure}
    \caption{Webové rozhraní: proces registrace a přihlášení}
\end{figure}

\begin{figure}[h]
    \centering
    \begin{subfigure}{0.48\textwidth}
        \centering
        \includegraphics[width=\linewidth]{../docs_server/screenshots/email_ver.png}
        \caption{Verifikace e-mailu}
        \label{fig:email_ver}
    \end{subfigure}
    \hfill
    \begin{subfigure}{0.48\textwidth}
        \centering
        \includegraphics[width=\linewidth]{../docs_server/screenshots/alerts_modal.png}
        \caption{Bezpečnostní upozornění (modal)}
        \label{fig:alerts_modal}
    \end{subfigure}
    \caption{Webové rozhraní: verifikace e-mailu a bezpečnostní upozornění}
\end{figure}

\begin{figure}[h]
    \centering
    \includegraphics[width=1\textwidth]{../docs_server/schemas/sequence_diagram_tracking.png}
    \caption{Sekvenční diagram: tok sledování zařízení (handshake, input, aktualizace polohy)}
    \label{fig:seq_tracking}
\end{figure}

\begin{figure}[h]
    \centering
    \includegraphics[width=1\textwidth]{../docs_server/schemas/sequence_dia_registration_hw.png}
    \caption{Sekvenční diagram: registrace HW zařízení (ESP32) do systému}
    \label{fig:seq_reg_hw}
\end{figure}

\begin{figure}[h]
    \centering
    \includegraphics[width=1\textwidth]{../docs_server/schemas/sequence_dia_registration_apk.png}
    \caption{Sekvenční diagram: registrace APK klienta (Android) a navázání session}
    \label{fig:seq_reg_apk}
\end{figure}

\begin{figure}[h]
    \centering
    \includegraphics[width=1\textwidth]{../docs_server/schemas/Use_case_diagram.png}
    \caption{Use-case diagram systému LOTR: aktéři a hlavní případy použití}
    \label{fig:use_case}
\end{figure}
\chapter{Aplikace pro Android}

\section{Koncept a cíle aplikace}
\label{sec:apk_koncept}
\subsection{Náhrada hardwarového trackeru}
\subsection{Princip Offline-First a odolnost proti výpadkům}

\section{TEORETICKÁ VÝCHODISKA A POUŽITÉ TECHNOLOGIE}
\label{sec:apk_technologie}
Vývoj pro mobilní platformu Android s sebou nese specifické výzvy, zejména v oblasti správy životního cyklu aplikací a úspory energie. Volba správných nástrojů je proto klíčová pro stabilitu výsledného řešení.

\subsection{Jazyk Kotlin a asynchronní programování (Coroutines)}
\textbf{Kotlin} je staticky typovaný programovací jazyk běžící na virtuálním stroji Java (JVM). Od roku 2019 je společností Google doporučován jako preferovaný jazyk pro vývoj Android aplikací. Oproti starší Javě přináší moderní prvky, jako je \textit{Null Safety} (typový systém odlišuje hodnoty, které mohou být null), což eliminuje časté pády aplikace na chybu \texttt{NullPointerException}.
Pro řešení asynchronních operací (síťová volání, databázové dotazy) Kotlin využívá koncept \textbf{Coroutines}. Na rozdíl od klasických vláken (Threads), která jsou náročná na systémové zdroje, jsou korutiny "lehká vlákna". Umožňují psát asynchronní kód sekvenčním stylem (pomocí klíčového slova \texttt{suspend}), což výrazně zvyšuje čitelnost a usnadňuje správu chyb.

\subsection{Systémové komponenty (Foreground Service, WorkManager)}
Operační systém Android agresivně ukončuje aplikace běžící na pozadí, aby šetřil baterii. Pro aplikace typu tracker je však nutné běžet nepřetržitě.
\begin{itemize}
    \item \textbf{Foreground Service (Služba na popředí):} Jedná se o službu, která dává systému najevo, že vykonává pro uživatele důležitou činnost. Musí zobrazovat trvalou notifikaci ve stavovém řádku. Díky tomu ji systém neukončí ani při nedostatku paměti.
    \item \textbf{WorkManager:} Pro úlohy, které nejsou okamžité, ale musí se zaručeně provést (např. odeslání dat na server, až bude dostupný internet), se využívá knihovna WorkManager. Ta inteligentně plánuje spuštění úloh s ohledem na stav baterie a sítě.
\end{itemize}

\subsection{Lokální persistence (Room Database)}
Pro ukládání strukturovaných dat přímo v zařízení se využívá databáze \textbf{SQLite}, která je součástí Androidu. Práce s ní pomocí surových SQL dotazů je však náchylná k chybám.
Proto byla zvolena knihovna \textbf{Room}, která slouží jako abstrakční vrstva (ORM) nad SQLite. Umožňuje definovat databázové tabulky jako datové třídy (Entity) a přístup k nim definovat pomocí rozhraní (DAO - Data Access Object). Room automaticky kontroluje správnost SQL dotazů již při kompilaci aplikace.

\subsection{Zabezpečené úložiště (EncryptedSharedPreferences)}
Ukládání citlivých dat (přihlašovací tokeny, API klíče) do běžných textových souborů (SharedPreferences) představuje bezpečnostní riziko, zejména na zařízeních s root oprávněním (tzv. rootnutých).
Knihovna \textbf{EncryptedSharedPreferences} řeší tento problém tím, že data před uložením automaticky šifruje. Šifrovací klíče jsou uloženy v \textbf{Android Keystore System}, což je speciální hardwarově chráněné úložiště, ke kterému nemá přístup ani samotný operační systém, natož malware.

\section{Architektura aplikace}
\label{sec:apk_architektura}
\subsection{Vrstevnatá architektura a reaktivní model (StateFlow)}
\subsection{Vlastní implementace HTTP klienta}

\section{Implementace klíčových funkcí}
\label{sec:apk_implementace}
\subsection{Sběr polohy na pozadí (LocationService)}
\subsection{Dávková synchronizace dat (SyncWorker)}
\subsection{Řízení spotřeby a vzdálené vypínání (PowerController)}
\subsection{Handshake protokol a konfigurace}

\section{Uživatelské rozhraní}
\label{sec:apk_ui}
\subsection{Hlavní ovládací panel (Dashboard)}
\subsection{Diagnostická konzole pro terénní testování}

	
\chapter*{Závěr a zhodnocení}
\noindent Seznam nedokonalostí či nedodělků je docela dlouhý. Jako hlavní nedostatek vidím absenci 3D tisknutého pouzdra pro HW tracker. V počátcích vývoje jsem sice vymodeloval prototyp, ale ukázalo se, že s tím jak byly postupně přidávány prvky nebo měněny stávající, nestačil model udržovat krok, jediná použitelná část zůstal držák na baterii + její nabíječku, ten se ukázal jako užitečný.
Dále celý základní princip API a identifikace mezi zařízeními a serverem je dosti primitivní a náchylný k chybám. V budoucnu by bylo vhodné přejít na nějaký propracovanější systém autentizace a autorizace zařízení, místo pouhého ID a uživatelských údajů ( je to takto dosti jednoduché na falšování požadavků).
Také možnosti řazení nebo třídění dat v uživatelském rozhraní chybí, a tak bych mohl pokračovat\dots

\noindent Co však dokončeno bylo, alespoň základně či přízemně, je samotné jadro systému. I když bez pokročilého zabezpečení tak API funguje a zařízení dokáží se serverem komunikovat a mají schopnosti se tak či onak vypořádat s vypadky sítě (cachování) a HW-tracker přestože s obtížemi by přece jenom šlo naložit do auta a sledovat jeho pohyb do doby vybití baterie (přes snahy o nízkou spotřeby však výdrž není úplně optimální). Nejlépe asi použít jako GPS alarm, kdy by se tracker hlasil s periodou cca. 1 dne a v případě změny by uivatel dostal hlášení. 
\noindent Celkově tedy systém funguje v urovni možností a schopností prototypu, ale pro reálné nasazení by bylo potřeba ještě hodně práce a vylepšení.
\addcontentsline{toc}{chapter}{Závěr}

\chapter*{Seznam použitých informačních zdrojů}
\addcontentsline{toc}{chapter}{Seznam použitých informačních zdrojů}
	
	%% literatura
	\begin{thebibliography}{99}
        % Hardware
		\bibitem{espressif_docs} ESPRESSIF SYSTEMS. \textit{ESP32 Technical Reference Manual} [Online]. 2024. Dostupné z: \url{https://www.espressif.com/en/support/documents/technical-documents}
		\bibitem{simcom_at} SIMCOM WIRELESS SOLUTIONS. \textit{A7600 Series AT Command Manual} [Online]. V1.01, 2021.
		\bibitem{ublox_datasheet} U-BLOX AG. \textit{NEO-6 u-blox 6 GPS Modules Data Sheet} [Online]. 2011. Dostupné z: \url{https://www.u-blox.com}
		\bibitem{freertos_guide} AMAZON WEB SERVICES. \textit{FreeRTOS Kernel Developer Guide} [Online]. Dostupné z: \url{https://www.freertos.org}
		\bibitem{tinygsm_repo} SHYMANSKYY, Volodymyr. \textit{TinyGSM: A small Arduino library for GSM modules} [Software]. GitHub, 2023. Dostupné z: \url{https://github.com/vshymanskyy/TinyGSM}

        % Server
		\bibitem{nodejs_docs} OPENJS FOUNDATION. \textit{Node.js Documentation} [Online]. Dostupné z: \url{https://nodejs.org/en/docs/}
		\bibitem{express_docs} EXPRESS. \textit{Express - Node.js web application framework} [Online]. Dostupné z: \url{https://expressjs.com}
		\bibitem{sequelize_manual} SEQUELIZE. \textit{Sequelize v6 Reference Manual} [Online]. Dostupné z: \url{https://sequelize.org}
		\bibitem{leaflet_docs} AGAFONKIN, Vladimir. \textit{Leaflet: an open-source JavaScript library for mobile-friendly interactive maps} [Online]. Dostupné z: \url{https://leafletjs.com}

        % Aplikace
		\bibitem{kotlin_docs} JETBRAINS. \textit{Kotlin Documentation} [Online]. Dostupné z: \url{https://kotlinlang.org/docs/home.html}
		\bibitem{android_arch} GOOGLE DEVELOPERS. \textit{Android Developers: Guide to App Architecture} [Online]. Dostupné z: \url{https://developer.android.com/jetpack/guide}
		\bibitem{android_workmanager} GOOGLE DEVELOPERS. \textit{Android WorkManager} [Online]. Dostupné z: \url{https://developer.android.com/topic/libraries/architecture/workmanager}

        % Ostatní
		\bibitem{sablonaSOC} DOKULIL Jakub. \textit{Šablona pro psaní SOČ v programu \LaTeX} [Online]. Brno, 2020. Dostupné z: \url{https://github.com/Kubiczek36/SOC_sablona}
        \bibitem{gemini_model} GOOGLE DEEPMIND. \textit{Gemini 3: Multimodal Generative AI Model} [Online]. 2025. Asistoval při generování a strukturování dokumentace.
	\end{thebibliography}
	
	%% obrázky 
	\listoffigures
	\addcontentsline{toc}{chapter}{Seznam obrázků}
	
	%% tabulky
	\listoftables
	\addcontentsline{toc}{chapter}{Seznam tabulek}
	
	\appendix %% začínají přílohy
	\addcontentsline{toc}{chapter}{Přílohy}
	
	\titleformat{\chapter}[block]{\scshape\bfseries\LARGE}{Příloha \thechapter}{10pt}{\vspace{0pt}}[\vspace{-22pt}] %% nastavení nadpisu u příloh

	
\end{document}