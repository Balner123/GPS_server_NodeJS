\chapter{Serverová část}

\section{Úvod a koncepce systému}
\label{sec:server_uvod}
Server zastává centrální roli v projektu. Zajišťuje komunikaci s hardwarovými jednotkami, trvalé ukládání telemetrických dat, správu uživatelských účtů a poskytování uživatelského rozhraní pro vizualizaci polohy a stavu zařízení, i jejich konfiguraci.

\noindent Hlavním úkolem serveru je agregace dat z jednotlivých GPS trackerů. Hardwarové jednotky odesílají data (poloha, síla signálu) prostřednictvím mobilní sítě na definované API endpointy. Server tato data validuje, zpracovává a ukládá do relační databáze.
Druhým klíčovým úkolem je obsluha klientských požadavků. Uživatelé přistupují k systému prostřednictvím webového prohlížeče. Server zajišťuje autentizaci uživatelů a generuje dynamické HTML stránky zobrazující mapové podklady s aktuální polohou sledovaných objektů.

\subsection{Monolitická architektura a MVC vzor}
Aplikace je navržena jako monolit, což znamená, že backendová logika i prezentační vrstva (frontend) jsou součástí jednoho projektu a běží v rámci jednoho procesu. Pro organizaci kódu byl zvolen architektonický vzor \textbf{MVC (Model-View-Controller)}, který odděluje data, logiku a zobrazení:

\begin{itemize}
    \item \textbf{Model (models/):} Definuje strukturu dat a logiku pro přístup k databázi. V našem případě je tato vrstva realizována pomocí ORM knihovny Sequelize (viz sekce \ref{subsec:server_db}). Modely odpovídají tabulkám v databázi (např. \texttt{User}, \texttt{Device}, \texttt{Telemetry}).
    \item \textbf{View (views/):} Stará se o prezentaci dat uživateli. Využíváme šablonovací systém EJS, který umožňuje vkládat data z backendu přímo do HTML struktury.
    \item \textbf{Controller (controllers/):} Přijímá požadavky od uživatele (nebo API), zpracovává je s využitím Modelů a rozhoduje, jaký Pohled se má zobrazit, nebo jaká data se mají vrátit (v případě JSON API).
\end{itemize}

Tento přístup umožňuje lepší organizaci kódu a rozšiřování o nové funkce.

\subsection{Relační databáze a ORM (MySQL + Sequelize)}
Jako databázový systém bylo zvoleno relační \textbf{MySQL}. Pro interakci s databází je využita technika \textbf{ORM} (Object-Relational Mapping) prostřednictvím knihovny Sequelize. Ta mapuje databázové tabulky na JavaScriptové objekty, což zjednodušuje manipulaci s daty.

\subsection{Principy autentizace a OAuth 2.0}
Pro zabezpečení přístupu a možností autentizace třetích stran je implementován mechanismus využívající knihovnu \texttt{passport.js}. Kromě lokálního přihlášení (email/heslo) je tak podporován i standard \textbf{OAuth 2.0} pro přihlášení přes externí poskytovatele (v našem případě: Google, GitHub).

\subsection{REST API architektura}
\textbf{REST} (Representational State Transfer) je architektonický styl pro návrh síťových aplikací. REST API definuje sadu pravidel pro komunikaci mezi klientem a serverem:
\begin{itemize}
    \item \textbf{Bezstavovost (Stateless):} Server neuchovává stav klienta mezi požadavky. Každý požadavek musí obsahovat všechny potřebné informace, ale v našem případě je toto použito pouze pro komunikaci s HW. Web a APK klienti využívají \textbf{sessions}.
    \item \textbf{Jednotné rozhraní:} Zdroje (data) jsou identifikovány pomocí URL a manipuluje se s nimi pomocí standardních HTTP metod (GET pro čtení, POST pro vytvoření, PUT pro úpravu, DELETE pro smazání).
\end{itemize}

\section{Návrh a implementace backendu}
\label{sec:server_backend}

\subsection{Databázová vrstva (MySQL a Sequelize)}
\label{subsec:server_db}
Databázová vrstva je postavena na relační databázi MySQL. Pro komunikaci s databází je využita knihovna \texttt{mysql2} ve spojení s ORM frameworkem Sequelize.
Pro iniciální konfiguraci databáze používáme soubor \textbf{init-db.sql}, který vytvoří potřebné tabulky a indexy.

\subsubsection{ORM Sequelize a definice modelů}
Sequelize abstrahuje SQL dotazy do JavaScriptových objektů. V projektu jsou definovány následující klíčové modely:
\begin{itemize}
      \setlength\itemsep{0.2em}
  \setlength\parskip{0pt}
  \setlength\parsep{0pt}
    \item \texttt{User}: Ukládá informace o uživatelích (jméno, email, hash hesla, OAuth ID).
    \item \texttt{Device}: Reprezentuje hardwarové jednotky (tracker). Obsahuje unikátní identifikátor, název a vazbu na vlastníka (\texttt{User}).
    \item \texttt{Location}: Uchovává historická data o poloze (zeměpisná šířka, délka, čas, rychlost).
    \item \texttt{Alert}: Záznamy o bezpečnostních událostech (např. opuštění geofence zóny nebo návrat do ní).
\end{itemize}
Vztahy mezi modely jsou definovány pouze jako 1:N (jeden uživatel má více zařízení, jedno zařízení má více záznamů polohy).
\begin{figure}[h]
    \centering
    \includegraphics[width=1\textwidth]{../docs_server/db_diagrams/DB_scheme.png} 
    \caption{ER diagram databázových modelů}
    \label{fig:db_schema}
\end{figure}

\subsubsection{API - HW tracker (ESP32)}
\label{sec:server_api}
HW klient se při komunikaci autentizuje pouze pomocí \texttt{device\_id} (získaného z MAC adresy), které je zaregistrováno, pokud uživatel vloží správné uživatelské údaje v registrační části OTA režimu. Nevyužívá session/cookie.

\begin{table}[H]
\centering
\begin{tabularx}{\textwidth}{|l|l|>{\raggedright\arraybackslash}X|}
\hline
Endpoint & Metoda & Popis \\ \hline
\texttt{/api/devices/register} & POST & Registrace HW zařízení k účtu. Payload: \texttt{client\_type="HW"}, \texttt{device\_id}, \texttt{username}, \texttt{password}, volitelné \texttt{name}. \\ \hline
\texttt{/api/devices/handshake} & POST & Periodická synchronizace stavu a konfigurace. Payload: \texttt{device\_id}, \texttt{client\_type="HW"}, \texttt{power\_status}. Odezva obsahuje \texttt{registered}, \texttt{config}, \texttt{power\_instruction}. \\ \hline
\texttt{/api/devices/input} & POST & Příjem telemetrie. Payload: jeden objekt nebo pole objektů se souřadnicemi, časem, rychlostí, \texttt{power\_status}. \\ \hline
\end{tabularx}
\caption{HW API endpointy}
\label{tab:api_hw}
\end{table}

\begin{enumerate}
      \setlength\itemsep{0.2em}
  \setlength\parskip{0pt}
  \setlength\parsep{0pt}
    \item Každý záznam musí obsahovat \texttt{device}, \texttt{latitude}, \texttt{longitude}, \texttt{timestamp}, může obsahovat \texttt{power\_status}.
    \item Server uloží lokace, aktualizuje \texttt{last\_seen}; pokud \texttt{power\_status} potvrzuje instrukci, vynuluje \texttt{power\_instruction}.
    \item Odezva \texttt{"success": true}; až poté klient smaže data z bufferu.
\end{enumerate}

Chybové stavy (HW):
\begin{itemize}
  \setlength\itemsep{0.2em}
  \setlength\parskip{0pt}
  \setlength\parsep{0pt}
    \item \texttt{404 / registered=false}: zařízení není známo; má přejít do konfiguračního režimu.
    \item \texttt{409}: \texttt{device\_id} patří jinému účtu; zařízení musí zastavit akce.
    \item \texttt{500+}: klient retry s backoff; data v bufferu se nemažou.
\end{itemize}

\subsubsection{API - APK klient (Android)}
APK klient používá uživatelské přihlášení (session cookie) a odděleně identifikuje zařízení pomocí \texttt{device\_id = installationId}. Telemetrie i handshake sdílí endpointy s HW, ale autentizační vrstva je odlišná.

\begin{table}[H]
\centering
\begin{tabularx}{\textwidth}{|l|l|>{\raggedright\arraybackslash}X|}
\hline
Endpoint & Metoda & Popis \\ \hline
    \texttt{/api/apk/login} & POST & Přihlášení uživatele; vyžaduje ověřený účet (\texttt{user.is\_verified=true}); vydá HTTP-only cookie \texttt{connect.sid}. \\ \hline
	\texttt{/api/apk/logout} & POST & Zrušení session; odhlášení APK klienta. \\ \hline
	\texttt{/api/devices/register} & POST & Registrace zařízení typu APK. Payload: \texttt{client\_type="APK"}, \texttt{device\_id} (UUID), \texttt{name}; vyžaduje platnou session. \\ \hline
	\texttt{/api/devices/handshake} & POST & Synchronizace konfigurace a power instrukcí. Payload: \texttt{device\_id}, \texttt{client\_type="APK"}, \texttt{power\_status}, \texttt{reason}. \\ \hline
	\texttt{/api/devices/input} & POST & Odeslání telemetrie z APK (batch array). Stejný formát jako HW, navíc může obsahovat \texttt{accuracy}. \\ \hline
\end{tabularx}
\caption{APK API endpointy}
\label{tab:api_apk}
\end{table}

APK Handshake (stejný endpoint, odlišný kontext): APK spouští handshake při startu služby, periodicky (např. 15 min) a po odeslání dávky. Server vrací \texttt{registered}, \texttt{config} a \texttt{power\_instruction}; APK musí respektovat \texttt{TURN\_OFF} zastavením služby.

APK Input: Telemetrie se odesílá dávkově z lokální SQLite (store-and-forward). Každý záznam obsahuje \texttt{device}, \texttt{latitude}, \texttt{longitude}, \texttt{timestamp}, \texttt{power\_status}, případně \texttt{speed} a \texttt{accuracy}. Server po \texttt{success:true} dovolí klientovi smazat dávku.

Autentizace a chyby (APK):
\begin{itemize}
      \setlength\itemsep{0.2em}
  \setlength\parskip{0pt}
  \setlength\parsep{0pt}
        \item \texttt{401/403}: neplatná nebo chybějící session; APK musí zneplatnit lokální session (logout broadcast).
        \item \texttt{404 registered=false}: zařízení není registrováno; klient zastaví službu a vyžádá novou registraci.
        \item \texttt{500+}: retry s exponential backoff; data v lokální DB zůstávají.
\end{itemize}
\clearpage

\subsubsection{Validace vstupních dat (Express Validator)}
Pro všechny vstupy (HW i APK) platí validace přes \texttt{express-validator}. Chybné payloady jsou odmítnuty kódem 400 ještě před zpracováním.

\begin{figure}[h]
    \centering
    \begin{subfigure}{0.48\textwidth}
        \centering
        \includegraphics[width=\linewidth]{../docs_server/screenshots/HW_server_respo.png}
        \caption{HW Handshake tok}
        \label{fig:hw_handshake_flow}
    \end{subfigure}
    \hfill
    \begin{subfigure}{0.48\textwidth}
        \centering
        \includegraphics[width=\linewidth]{../docs_server/screenshots/HW_APK_input.png}
        \caption{HW/APK Input tok}
        \label{fig:hw_apk_input_flow}
    \end{subfigure}
\end{figure}

\subsubsection{Dokumentace API (Swagger)}
Swagger (\texttt{swagger-jsdoc} + \texttt{swagger-ui-express}) generuje strojově čitelnou dokumentaci z komentářů controllerů a je dostupný na \texttt{/api-docs}.

\begin{figure}[h]
    \centering
    \includegraphics[width=0.9\textwidth]{../docs_server/screenshots/swagger_docs.png} 
    \caption{Ukázka Swagger dokumentace API}
    \label{fig:swagger_api}
\end{figure}

\clearpage

\subsection{Správa uživatelů a Autorizace}
\label{subsec:server_users}
Používáme uživatelské účty s možností registrace, přihlášení a správy jejich údajů. Pro bezpečnou autentizaci a autorizaci je implementován middleware \texttt{authorization.js} (popsáno výše).

\subsubsection{Autorizace a role (authorization.js)}
Middleware \texttt{authorization.js} vynucuje přihlášení, odděluje běžné uživatele, administrátory a blokuje nevhodné akce ROOT účtu, a zároveň autentizuje HW/APK zařízení na základě \texttt{device\_id}:
\begin{itemize}
    \setlength\itemsep{0.2em}
    \item \texttt{isAuthenticated}/\texttt{isApiAuthenticated}: vyžadují platnou session pro web i API.
    \item \texttt{isUser}/\texttt{isRoot}/\texttt{isNotRootApi}: směrují podle role, zakazují ROOT na běžných API a naopak.
    \item \texttt{authenticateDevice}: pro HW/APK čte \texttt{device\_id}, ověří vazbu na \texttt{User} a naplní \texttt{req.device}, \texttt{req.user}, \texttt{req.clientType}.
\end{itemize}

\begin{table}[H]
\centering
\begin{tabularx}{\textwidth}{|l|>{\raggedright\arraybackslash}X|>{\raggedright\arraybackslash}X|}
\hline
Role / middleware & Webové routy & API routy \\ \hline
	\texttt{isAuthenticated} & Zobrazí HTML nebo přesměruje na \texttt{/login}. & Vrací 401 JSON, neprovádí přesměrování. \\ \hline
	\texttt{isUser} & Povolen běžný uživatel, \texttt{root} přesměrován do administrace. & Kombinuje se s \texttt{isNotRootApi} pro blokaci \texttt{root} na uživatelských API. \\ \hline
	\texttt{isRoot} & Otevírá \texttt{/administration}. & Chrání \texttt{/api/admin/*}. \\ \hline
	\texttt{authenticateDevice} & N/A & Povinné pro \texttt{/api/devices/input}, \texttt{/api/devices/handshake} váže \texttt{device\_id} k uživateli. \\ \hline
\end{tabularx}
\caption{Rychlá orientace v rolích a middleware}
\label{tab:auth_roles}
\end{table}


\begin{minted}{javascript}
    const isAuthenticated = (req, res, next) => {
  if (hasUserSession(req)) {
    return next();
  }
  req.flash?.('error', 'Please log in to view this page.');
  res.redirect('/login');
};
\end{minted}

\subsubsection{Sessions}
Relace (sessions) jsou spravovány pomocí \texttt{express-session}. Klíč \texttt{SESSION\_SECRET} se načítá z prostředí, cookie je \texttt{httpOnly} s \texttt{maxAge} 6 hodin. Flag \texttt{secure} se zapíná, pokud je \texttt{NODE\_ENV=using\_ssl}, aby se cookie přenášela jen přes HTTPS. Webové routy používají \texttt{isAuthenticated} a případně přesměrování, API vrací JSON přes \texttt{isApiAuthenticated}.

\subsubsection{Správa identit (Passport.js)}
Knihovna \texttt{passport.js} zajišťuje flexibilní autentizaci. V systému jsou implementovány tři strategie:
\begin{itemize}
      \setlength\itemsep{0.2em}
  \setlength\parskip{0pt}
  \setlength\parsep{0pt}
    \item \textbf{Local Strategy:} Přihlášení pomocí emailu a hesla.
    \item \textbf{Google Strategy:} OAuth 2.0 přihlášení přes Google účet (\texttt{passport-google-oauth20}).
    \item \textbf{GitHub Strategy:} OAuth 2.0 přihlášení přes GitHub účet (\texttt{passport-github2}).
\end{itemize}
Po úspěšném přihlášení je uživatelská relace (session) uložena a identifikována pomocí cookie.

\begin{minted}{javascript}
passport.use(new GoogleStrategy({
  clientID: process.env.GOOGLE_CLIENT_ID,
  clientSecret: process.env.GOOGLE_CLIENT_SECRET,
  callbackURL: process.env.GOOGLE_CALLBACK_URL,
  scope: ['profile', 'email']
}, async (accessToken, refreshToken, profile, done) => {...}));

passport.use(new GitHubStrategy({
  clientID: process.env.GITHUB_CLIENT_ID,
  clientSecret: process.env.GITHUB_CLIENT_SECRET,
  callbackURL: process.env.GITHUB_CALLBACK_URL,
  scope: ['user:email'] // Important to get the email
}, async (accessToken, refreshToken, profile, done) => {...}));
\end{minted}


\subsubsection{Hashování hesel (Bcrypt.js)}
Hesla uživatelů nejsou nikdy ukládána v otevřené podobě. Při registraci je heslo prohnáno hashovací funkcí \texttt{bcrypt} se solí (salt). Při přihlášení se zadané heslo zahashuje a porovná s uloženým hashem. To chrání uživatele i v případě úniku databáze.


\subsubsection{ROOT účet} Z provozních důvodů systém startuje s hardcoded administrátorským účtem \texttt{ROOT}. Tento účet má vlastní administraci, která umožňuje přímý přístup do DB a manipulaci s jejími daty.

\begin{figure}[h]
    \centering
    \includegraphics[width=1\textwidth]{../docs_server/screenshots/root.png} 
    \caption{Administrátorský panel}
    \label{fig:admin_panel}
\end{figure}

\subsubsection{E-mailová komunikace (Nodemailer)}
Pro interakci s uživatelem mimo webové rozhraní slouží knihovna \texttt{nodemailer}. Využívá se v následujících scénářích:
\begin{itemize}
          \setlength\itemsep{0.2em}
  \setlength\parskip{0pt}
  \setlength\parsep{0pt}
    \item \textbf{Verifikace emailu:} Po registraci je odeslán unikátní kód pro ověření existence emailové schránky.
    \item \textbf{Reset hesla:} Odeslání odkazu pro obnovu zapomenutého hesla.
    \item \textbf{Bezpečnostní alerty:} Upozornění uživatele, pokud jeho zařízení opustí nastavenou bezpečnou zónu (Geofence).
\end{itemize}

\begin{figure}[h]
    \centering
    \begin{subfigure}{0.48\textwidth}
        \centering
        \includegraphics[width=\linewidth]{../docs_server/screenshots/nodemailer_trans.png}
        \caption{Použití Nodemailer.js}
        \label{fig:nodemailer_email}
    \end{subfigure}
    \hfill
    \begin{subfigure}{0.48\textwidth}
        \centering
        \includegraphics[width=\linewidth]{../docs_server/screenshots/warning_email.png}
        \caption{Upozornění na opuštění geofence zóny}
        \label{fig:geofence_email}
    \end{subfigure}
    \caption{E-mailové scénáře: transakční zprávy a bezpečnostní upozornění}
\end{figure}

\clearpage
\section{Prezentační vrstva (Frontend)}
\label{sec:server_frontend}
Frontendová část aplikace je navržena s myšlenkou jednoduchosti. Kombinuje "server-side rendering" (EJS) pro základní strukturu stránky a klientský JavaScript pro dynamické aktualizace dat v reálném čase.

\subsection{Šablonovací systém EJS a struktura pohledů}
\label{subsec:server_ejs}
Pro generování HTML stránek na straně serveru je použit šablonovací systém \textbf{EJS (Embedded JavaScript)}. Ten umožňuje vkládat JavaScriptovou logiku přímo do HTML kódu.
Struktura pohledů je modularizována pomocí tzv. \textit{partials} (dílčích šablon), což zabraňuje duplicitě kódu. Typická stránka se skládá z:
\begin{itemize}
  \setlength\itemsep{0.2em}
  \setlength\parskip{0pt}
  \setlength\parsep{0pt}
    \item \texttt{\_head.ejs}: Meta tagy, importy CSS stylů a externích knihoven.
    \item \texttt{\_navbar.ejs}: Navigační lišta s odkazy a informacemi o přihlášeném uživateli.
    \item \textbf{Obsah stránky:} Unikátní obsah pro daný pohled (např. \texttt{index.ejs} pro mapu, \texttt{settings.ejs} pro nastavení).
    \item \texttt{\_footer.ejs}: Patička stránky a importy JavaScriptových souborů.
\end{itemize}
Data z backendu (např. seznam zařízení, chybové hlášky) jsou do šablon předávána při vykreslování v controlleru.

\subsection{Vizualizace dat a mapové podklady}
\label{subsec:server_maps}
Klíčovou funkcí frontendu je vizualizace polohy trackerů na mapě. Pro tento účel byla zvolena open-source knihovna \textbf{Leaflet.js}, která je lehká a flexibilní.
Jako zdroj mapových podkladů (dlaždic) slouží služba \textbf{OpenStreetMap}.

\subsubsection{Dynamická aktualizace (AJAX)}
Aby uživatel viděl aktuální polohu zařízení bez nutnosti obnovovat celou stránku, využívá aplikace technologii \textbf{AJAX} (Asynchronous JavaScript and XML) prostřednictvím moderního Fetch API.

\begin{enumerate}  \setlength\itemsep{0.2em}
  \setlength\parskip{0pt}
  \setlength\parsep{0pt}

    \item Při načtení stránky se inicializuje mapa a vykreslí se poslední známé polohy.
    \item Klientský skript (\texttt{index.js}) spouští v pravidelných intervalech (nastaveno na 5 sekund) požadavek na API endpoint \texttt{/api/devices/coordinates}.
    \item Server vrátí aktuální data ve formátu JSON.
    \item JavaScript na klientovi porovná nová data s existujícími markery na mapě a aktualizuje jejich pozici, případně obsah informačních bublin (tooltipů).
\end{enumerate}

\begin{figure}[h]
    \centering
    \includegraphics[width=0.9\textwidth]{../docs_server/screenshots/manage.png}
    \caption{Webové rozhraní - Správa zařízení}
    \label{fig:web_manage}
\end{figure}

\begin{figure}[h]
    \centering
    \includegraphics[width=0.9\textwidth]{../docs_server/screenshots/settings.png}
    \caption{Webové rozhraní - Settings}
    \label{fig:web_settings}
\end{figure}

\section{Nasazení - Docker}
\label{subsec:server_docker}
Pro zajištění snadného nasazení na různé servery je celý systém kontejnerizován pomocí technologie \textbf{Docker}. 
Obraz (image) serverové aplikace je definován v souboru \texttt{Dockerfile}. Jako základ je použit oficiální odlehčený obraz \texttt{node:24-slim}, který minimalizuje velikost výsledného kontejneru.
Proces sestavení obrazu zahrnuje:
\begin{enumerate}
  \setlength\itemsep{0.2em}
  \setlength\parskip{0pt}
  \setlength\parsep{0pt}
    \item Nastavení pracovního adresáře na \texttt{/app}.
    \item Kopírování definice závislostí (\texttt{package.json}) a jejich instalace pomocí \texttt{npm install}.
    \item Kopírování zdrojových kódů aplikace.
    \item Expozice portu 5000, na kterém server naslouchá.
    \item Definice spouštěcího příkazu \texttt{node server.js}.
\end{enumerate}
\begin{minted}{dockerfile}
FROM node:24-slim
WORKDIR /app
COPY package*.json ./
RUN npm install
COPY . .
EXPOSE 5000
CMD ["node", "server.js"]
\end{minted}

\subsection{Orchestrace - Docker Compose}
\label{subsec:server_compose}
Protože systém vyžaduje ke svému běhu nejen aplikační server, ale i databázi, využíváme \textbf{Docker Compose} pro definici a spouštění více kontejnerů současně. 
Konfigurace je uložena v souboru \texttt{docker-compose.yml}, který definuje dvě hlavní služby:

\begin{itemize}
    \item \textbf{app:} Samotná Node.js aplikace. Služba je nakonfigurována tak, aby čekala na plné spuštění databáze (\texttt{depends\_on}).
    \item \textbf{mysql:} Databázový server MySQL verze 8.0. Data jsou ukládána do trvalého svazku (volume) \texttt{mysql-data}, což zajišťuje, 
    že data přežijí i restart nebo smazání kontejneru. Při prvním spuštění se automaticky provede inicializační skript \texttt{init-db.sql}.
\end{itemize}

Obě služby běží ve společné virtuální síti \texttt{gps-network}, což jim umožňuje vzájemnou komunikaci pomocí názvů služeb (hostname).



\begin{figure}[H]
    \centering
    % Row 1
    \begin{subfigure}{0.3\textwidth}
        \centering
        \includegraphics[width=\linewidth]{../docs_server/screenshots/registration.png}
        \caption{Registrace}
        \label{fig:registration}
    \end{subfigure}
    \hfill
    \begin{subfigure}{0.3\textwidth}
        \centering
        \includegraphics[width=\linewidth]{../docs_server/screenshots/login.png}
        \caption{Přihlášení}
        \label{fig:login}
    \end{subfigure}

    \vspace{0.8em}

    % Row 2
    \begin{subfigure}{0.3\textwidth}
        \centering
        \includegraphics[width=\linewidth]{../docs_server/screenshots/email_ver.png}
        \caption{Verifikace e-mailu}
        \label{fig:email_ver}
    \end{subfigure}
    \hfill
    \begin{subfigure}{0.3\textwidth}
        \centering
        \includegraphics[width=\linewidth]{../docs_server/screenshots/alerts_modal.png}
        \caption{Bezpečnostní upozornění (modal)}
        \label{fig:alerts_modal}
    \end{subfigure}

    \caption{Webové rozhraní: registrace, přihlášení, verifikace e-mailu a bezpečnostní upozornění}
\end{figure}


\clearpage

\begin{figure}[h]
    \centering
    \includegraphics[width=1\textwidth]{../docs_server/schemas/Use_case_diagram.png}
    \caption{Use-case diagram systému LOTR: aktéři a hlavní případy použití}
    \label{fig:use_case}
\end{figure}

\begin{figure}[h]
    \centering
    \includegraphics[width=1\textwidth]{../docs_server/schemas/sequence_diagram_tracking.png}
    \caption{Sekvenční diagram: tok sledování zařízení (handshake, input, aktualizace polohy)}
    \label{fig:seq_tracking}
\end{figure}

