\chapter{Aplikace pro Android}

\section{Úvod a Koncept}
\label{sec:apk_koncept}
"Mobilní aplikace v systému LOTR plní roli softwarového trackeru". Avšak jak již bylo zmíněno v úvodu, tak aplikace plní roli doplňkovou a testovací, a popis jejího kodu je vytvořen polo-reverzním inženýsrtvím. 
Tedy: aplikace, ačkoliv sdílí stejný backend a datové formáty jako hardwarové jednotky, její vnitřní logika je odlišná kvůli specifikům operačního systému Android a jeho omezením.
Aplikace transformuje běžný smartphone na plnohodnotné sledovací zařízení. Využívá integrovaný GPS/GNSS přijímač a mobilní datové připojení. 
Na rozdíl od jednoúčelového HW trackeru, který běží na "bare metal" firmware, musí aplikace koexistovat s operačním systémem a ostatními aplikacemi, což přináší nutnost řešit priority procesů a správu paměti.

\subsection{Princip Offline-First a odolnost proti výpadkům}
Mobilní zařízení se často pohybují v oblastech s nestabilním signálem (tunely, podzemní garáže, venkov). Aby nedošlo ke ztrátě cenných telemetrických dat, je aplikace navržena podle principu \textbf{Offline-First}.
To znamená, že naměřená data nejsou nikdy odesílána přímo na server. Proces je následující:
\begin{enumerate}
      \setlength\itemsep{0.2em}
  \setlength\parskip{0pt}
  \setlength\parsep{0pt}
    \item Získání souřadnic z GPS modulu.
    \item Okamžité uložení do lokální databáze (SQLite/Room).
    \item Pokus o odeslání dávky dat na server (na pozadí).
    \item Po úspěšném potvrzení serverem jsou data z lokální databáze smazána.
\end{enumerate}
Tento přístup garantuje, že i při dlouhodobém výpadku konektivity zůstanou data bezpečně uložena v telefonu a odešlou se automaticky po obnovení spojení.

\subsection{Komplikace způsobené Androidem(Doze mode, Background limits)}
Moderní verze systému Android obsahují agresivní mechanismy pro úsporu energie, jako je \textbf{Doze mode} a \textbf{App Standby buckets}. Tyto systémy omezují přístup aplikací k síti a CPU, pokud běží na pozadí a uživatel s nimi neinteraguje.
Pro aplikaci typu tracker, která musí běžet nepřetržitě, to představuje zásadní problém. Řešením je využití tzv. \textbf{Foreground Service} (služby na popředí), která dává systému signál, že aplikace vykonává pro uživatele kritickou činnost, a nesmí být ukončena ani uspána (viz sekce \ref{sec:apk_background}).

\section{Architektura a použité technologie}
\label{sec:apk_architektura}
Aplikace je navržena modulárně s principem oddělení odpovědností ("Separation of Concerns"). Architektura respektuje moderní doporučení pro vývoj Android aplikací, avšak s ohledem na specifické potřeby trackeru (běh na pozadí) využívá i některé nízkoúrovňové mechanismy.

\subsection{Jazyk Kotlin a asynchronní programování (Coroutines)}
Jako primární jazyk byl zvolen \textbf{Kotlin}, který je od roku 2019 preferovaným jazykem pro Android. Jeho klíčovou vlastností je \textit{Null Safety}, která eliminuje celou třídu chyb typu \texttt{NullPointerException}.
Pro asynchronní operace (síťová komunikace, přístup k databázi) jsou využívány \textbf{Kotlin Coroutines}. Ty umožňují psát asynchronní kód sekvenčním způsobem bez nutnosti vnořování callbacků ("callback hell"). Klíčové slovo \texttt{suspend} označuje funkce, které mohou pozastavit vykonávání korutiny bez blokování hlavního vlákna.

\subsection{Vrstevnatá architektura (UI, Service, Data)}
Aplikace je rozdělena do tří logických vrstev:
\begin{itemize}
      \setlength\itemsep{0.2em}
  \setlength\parskip{0pt}
  \setlength\parsep{0pt}
    \item \textbf{UI Vrstva (Activity):} Zodpovídá pouze za vykreslování dat a reakci na vstupy uživatele. Neobsahuje žádnou aplikační logiku.
    \item \textbf{Service Vrstva (LocationService, SyncWorker):} Obsahuje "business logiku" aplikace. Běží nezávisle na UI a zajišťuje sběr dat a komunikaci se serverem.
    \item \textbf{Data Vrstva (Repository, Database):} Zajišťuje jednotný přístup k datům. \texttt{ServiceStateRepository} drží aktuální stav aplikace, zatímco \texttt{AppDatabase} (Room) slouží pro trvalé uložení telemetrie.
\end{itemize}

\subsection{Reaktivní řízení stavu (StateFlow)}
Pro komunikaci mezi službou běžící na pozadí a uživatelským rozhraním je využit reaktivní vzor pomocí \textbf{StateFlow}.
Třída \texttt{ServiceStateRepository} funguje jako Singleton a drží aktuální stav služby (např. \texttt{isRunning}, \texttt{gpsCount}, \texttt{lastLocation}).
\begin{itemize}
      \setlength\itemsep{0.2em}
  \setlength\parskip{0pt}
  \setlength\parsep{0pt}
    \item Služba (\texttt{LocationService}) aktualizuje stav pomocí \texttt{MutableStateFlow}.
    \item UI (\texttt{MainActivity}) odebírá (collect) změny stavu pomocí \texttt{asStateFlow()} a automaticky překresluje obrazovku při každé změně.
\end{itemize}
Tento přístup eliminuje potřebu zastaralých mechanismů jako \texttt{LocalBroadcastManager} a zajišťuje, že UI vždy zobrazuje aktuální data.

\subsection{Vlastní implementace HTTP klienta}
Na rozdíl od běžných aplikací, které využívají knihovny jako Retrofit, je v tomto projektu implementován vlastní HTTP klient (\texttt{ApiClient}) postavený přímo na třídě \texttt{HttpURLConnection}.
Důvody pro toto rozhodnutí jsou:
\begin{enumerate}
    \setlength\itemsep{0.2em}
  \setlength\parskip{0pt}
  \setlength\parsep{0pt}
    \item \textbf{Plná kontrola:} Přímá manipulace s cookies (pro session management) a timeouty bez nutnosti konfigurace složitých wrapperů.
    \item \textbf{Minimalismus:} Snížení velikosti výsledné aplikace odstraněním velkých externích závislostí.
\end{enumerate}

\begin{figure}[h]
    \centering
    \includegraphics[width=0.8\textwidth]{../docs_apk/screenshots/api_client.png}
    \caption{API client}
    \label{fig:api_client}
\end{figure}
\section{Implementace služeb na pozadí (Background Processing)}
\label{sec:apk_background}
Jádrem aplikace je schopnost spolehlivě běžet na pozadí, sbírat data o poloze a odesílat je na server, i když je telefon uzamčený v kapse uživatele. Toho je dosaženo kombinací \texttt{Foreground Service} a \texttt{WorkManageru}.

\subsection{Foreground Service (LocationService)}
Operační systém Android agresivně ukončuje aplikace běžící na pozadí, aby šetřil baterii. Pro aplikace typu tracker je však nutné běžet nepřetržitě.
Řešením je implementace tzv. \textbf{Foreground Service} (služba na popředí). Třída \texttt{LocationService} při svém startu vytvoří trvalou notifikaci v systémové liště (povinnost daná systémem Android 8.0+). Tím dává systému signál, že aplikace vykonává pro uživatele kritickou činnost a nesmí být ukončena ani při nedostatku paměti (OOM Killer).
Služba také monitoruje stav systémových providerů (GPS) pomocí \texttt{BroadcastReceiver}. Pokud uživatel vypne GPS v nastavení telefonu, služba to detekuje, automaticky se ukončí a zaloguje důvod vypnutí.

\begin{figure}[h]
    \centering
    \includegraphics[width=0.8\textwidth]{../docs_apk/screenshots/foreground_service.png}
    \caption{Foreground Service}
    \label{fig:foreground_service}
\end{figure}

\subsection{Sběr a filtrace polohy (FusedLocationProvider)}
Pro získávání souřadnic není využíváno surové GPS API, ale moderní \textbf{Fused Location Provider API} (součást Google Play Services). Toto API inteligentně kombinuje data z GPS, Wi-Fi, Bluetooth a mobilních vysílačů pro rychlejší a přesnější určení polohy s nižší spotřebou energie.
Získaná poloha je v callbacku \texttt{onLocationResult} okamžitě uložena do lokální databáze Room. Aplikace nečeká na úspěšné odeslání dat (princip Fire-and-Forget směrem do DB), což zajišťuje, že sběr dat není blokován pomalým síťovým připojením.

\subsection{Plánování úloh a synchronizace (WorkManager)}
Zatímco \texttt{LocationService} běží neustále, odesílání dat je řešeno periodicky pomocí knihovny \textbf{WorkManager}. Třída \texttt{SyncWorker} je naplánována ke spuštění každých 15 minut (nebo častěji při detekci pohybu).
Proces synchronizace:
\begin{enumerate}
        \setlength\itemsep{0.2em}
  \setlength\parskip{0pt}
  \setlength\parsep{0pt}
    \item Worker načte z databáze dávku (batch) 50 nejstarších neodeslaných záznamů.
    \item Data jsou serializována do JSON formátu a odeslána na endpoint \texttt{/api/devices/input}.
    \item Pokud server odpoví \texttt{200 OK} (nebo \texttt{201 Created}), worker smaže úspěšně odeslané záznamy z lokální databáze.
    \item Pokud odeslání selže (např. není signál), data zůstávají v DB a WorkManager automaticky naplánuje opakování s exponenciálním odkladem (Exponential Backoff).
\end{enumerate}

\begin{figure}
    \centering
    \includegraphics[width=1\textwidth]{../docs_apk/screenshots/json_payload.png}
    \caption{sestavení JSON payload pro odeslání polohy}
    \label{fig:json_payload}
\end{figure}

\subsection{Řízení spotřeby (PowerController)}
Aplikace implementuje logiku pro vzdálené vypnutí trackeru ze strany serveru. Třída \texttt{PowerController} funguje jako stavový automat. Pokud server v odpovědi na handshake pošle instrukci \texttt{TURN\_OFF}, aplikace:
\begin{enumerate}
        \setlength\itemsep{0.2em}
  \setlength\parskip{0pt}
  \setlength\parsep{0pt}
    \item Zastaví sběr polohy (\texttt{LocationService}).
    \item Uloží do preferencí příznak \texttt{isTurnOffAckPending}.
    \item Při příštím pokusu o spuštění (např. po restartu telefonu) \texttt{PowerController} zablokuje start služby, dokud není serveru potvrzeno, že příkaz k vypnutí byl přijat.
\end{enumerate}
Tato logika brání "zombie" chování, kdy by se vypnutý tracker neustále sám zapínal. Uživatel má však možnost v UI vynutit manuální spuštění (Manual Override).

\section{Správa dat a komunikace}
\label{sec:apk_data}
\subsection{Lokální persistence (Room Database)}
Pro ukládání strukturovaných dat přímo v zařízení se využívá databáze \textbf{SQLite}, která je součástí Androidu. Práce s ní pomocí surových SQL dotazů je však náchylná k chybám.
Proto byla zvolena knihovna \textbf{Room}, která slouží jako abstrakční vrstva (ORM) nad SQLite. Umožňuje definovat databázové tabulky jako datové třídy (Entity) a přístup k nim definovat pomocí rozhraní (DAO - Data Access Object). Room automaticky kontroluje správnost SQL dotazů již při kompilaci aplikace.
Entita \texttt{CachedLocation} reprezentuje jeden záznam o poloze a obsahuje kromě souřadnic i metadata jako přesnost měření (\texttt{accuracy}), počet satelitů a stav baterie.

\begin{figure}[h]
    \centering
    \includegraphics[width=0.8\textwidth]{../docs_apk/screenshots/db_room_entity.png}
    \caption{Room Database - CachedLocation entity}
    \label{fig:room_database}
\end{figure}

\subsection{Bezpečné úložiště (EncryptedSharedPreferences)}
Ukládání citlivých dat (přihlašovací tokeny, API klíče) do běžných textových souborů (SharedPreferences) představuje bezpečnostní riziko, zejména na zařízeních s root oprávněním (tzv. rootnutých).
Knihovna \textbf{EncryptedSharedPreferences} řeší tento problém tím, že data před uložením automaticky šifruje. Šifrovací klíče jsou uloženy v \textbf{Android Keystore System}, což je speciální hardwarově chráněné úložiště, ke kterému nemá přístup ani samotný operační systém, natož malware.
V aplikaci jsou takto ukládány následující údaje:
\begin{itemize}
          \setlength\itemsep{0.2em}
  \setlength\parskip{0pt}
  \setlength\parsep{0pt}
    \item \texttt{session\_cookie}: Autentizační token pro komunikaci s API.
    \item \texttt{installation\_id}: Unikátní identifikátor zařízení (generovaný jako hash z UUID).
    \item \texttt{server\_url}: Adresa backendu (umožňuje dynamickou změnu prostředí).
\end{itemize}

\subsection{HTTP Klient a komunikace s API}
Komunikace se serverem probíhá výhradně přes zabezpečený protokol HTTPS. Vlastní implementace klienta (\texttt{ApiClient}) zajišťuje:
\begin{itemize}
          \setlength\itemsep{0.2em}
  \setlength\parskip{0pt}
  \setlength\parsep{0pt}
    \item Přidávání hlavičky \texttt{Content-Type: application/json}.
    \item Správu cookies (odesílání session cookie s každým požadavkem).
    \item Ošetření síťových chyb a timeoutů (nastaveno na 15 sekund).
\end{itemize}
V případě chyby 401 (Unauthorized) klient automaticky vymaže neplatnou session a přepne aplikaci do stavu "odhlášeno", což uživatele vyzve k novému přihlášení.

\section{Uživatelské rozhraní (UI)}
\label{sec:apk_ui}
Uživatelské rozhraní je navrženo jako minimalistické a funkční, s důrazem na přehlednost stavu služby.

\subsection{Registrace a přihlášení (LoginActivity)}
Tato obrazovka slouží k prvotnímu spárování zařízení s uživatelským účtem. Uživatel zadává své přihlašovací údaje (stejné jako do webové aplikace).
Aplikace při prvním spuštění vygeneruje unikátní \texttt{installation\_id} (10 znaků), které se při přihlášení odešle na server. Tím dojde k vytvoření záznamu v tabulce \texttt{Devices}.
\textbf{Developer Mode:} Dlouhým stiskem na nadpis aplikace se zobrazí skryté pole pro změnu URL adresy serveru. To umožňuje snadné přepínání mezi vývojovým (localhost) a produkčním prostředím bez nutnosti rekompilace aplikace.

\begin{figure}
    \centering
    \includegraphics[width=0.8\textwidth]{../docs_apk/screenshots/installationID.png}
    \caption{\texttt{getInstallationId()}}
    \label{fig:installation_id}
\end{figure}

\subsection{Hlavní ovládací panel (Dashboard)}
Po přihlášení se zobrazí hlavní obrazovka (\texttt{MainActivity}), která poskytuje:
\begin{itemize}
          \setlength\itemsep{0.2em}
  \setlength\parskip{0pt}
  \setlength\parsep{0pt}
    \item \textbf{Indikátor stavu:} Velká ikona a text informující, zda služba běží, je zastavena, nebo čeká na GPS signál.
    \item \textbf{Ovládací prvky:} Tlačítka START a STOP pro manuální řízení služby.
    \item \textbf{Statistiky:} Počet nasbíraných bodů v lokální databázi a čas posledního úspěšného odeslání dat.
\end{itemize}
\begin{figure}[h]
    \centering
    \begin{minipage}[t]{0.48\textwidth}
        \centering
        \includegraphics[width=\linewidth]{../docs_apk/screenshots/ONstate.png}
        \caption*{Zapnuto}
    \end{minipage}\hfill
    \begin{minipage}[t]{0.48\textwidth}
        \centering
        \includegraphics[width=\linewidth]{../docs_apk/screenshots/offstate.png}
        \caption*{Vypnuto}
    \end{minipage}
    \caption{Hlavní ovládací panel}
    \label{fig:dashboard}
\end{figure}
\begin{figure}[h]
    \centering
    \begin{minipage}[t]{0.48\textwidth}
        \centering
        \includegraphics[width=\linewidth]{../docs_apk/screenshots/in_background_notification.png}
        \caption*{Na pozadí}
    \end{minipage}\hfill
    \begin{minipage}[t]{0.48\textwidth}
        \centering
        \includegraphics[width=\linewidth]{../docs_apk/screenshots/Offline_pending_data_notification.png}
        \caption*{Offline – čekající data}
    \end{minipage}
    \caption{Stav čekání na GPS signál}
    \label{fig:nogps_state}
\end{figure}

\subsection{Diagnostická nástroje a konzole}
Pro účely ladění v terénu obsahuje aplikace integrovanou konzoli (\texttt{ConsoleLogger}). Ta zobrazuje interní logy aplikace (např. "GPS fix acquired", "Upload failed: timeout") přímo na displeji telefonu.
Rozlišujeme 4 úrovně logování:
\begin{itemize}
        \setlength\itemsep{0.2em}
    \setlength\parskip{0pt}
    \setlength\parsep{0pt}
      \item \textbf{ERROR:} Kritické chyby, které vyžadují zásah uživatele.
      \item \textbf{WARNING:} Varování o potenciálních problémech.
      \item \textbf{INFO:} Obecné informace o běhu aplikace
      \item \textbf{DEBUG:} Detailní informace.
\end{itemize}
\begin{figure}[h]
    \centering
    \begin{minipage}[t]{0.48\textwidth}
      \centering
      \includegraphics[width=\linewidth]{../docs_apk/screenshots/console_debug.png}
      \caption*{DEBUG konsola}
    \end{minipage}\hfill
    \begin{minipage}[t]{0.48\textwidth}
      \centering
      \includegraphics[width=\linewidth]{../docs_apk/screenshots/4_state_Console_settings.png}
      \caption*{Nastavení úrovní logování}
    \end{minipage}
    \caption{Diagnostická konzole}
    \label{fig:console_logger}
\end{figure}


\begin{figure}[h]
    \centering
    \includegraphics[width=1\textwidth]{../docs_apk/schemas/use_case_diagram.png}
    \caption{Use Case diagram aplikace}
    \label{fig:use_case_diagram}
\end{figure}

\begin{figure}[h]
    \centering
    \includegraphics[width=0.5\textwidth]{../docs_apk/schemas/state_diagram.png}
    \caption{State diagram aplikace}
    \label{fig:state_diagram}
\end{figure}