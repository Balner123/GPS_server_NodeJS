\documentclass[12pt, a4paper]{report}

% --- Základní nastavení kódování a jazyka ---
\usepackage[utf8]{inputenc}
\usepackage[T1]{fontenc}
\usepackage[czech]{babel}

% --- Balíčky pro sazbu ---
\usepackage{graphicx} % Pro obrázky
\usepackage{amsmath} % Pro matematiku
\usepackage{geometry} % Pro nastavení okrajů
\geometry{
 left=30mm,
 right=25mm,
 top=30mm,
 bottom=30mm,
}

% --- Balíčky pro kód a odkazy ---
\usepackage{listings} % Pro výpisy kódu
\usepackage{xcolor} % Pro barvy v kódu
\usepackage{hyperref} % Pro klikatelné odkazy
\usepackage{tocbibind} % Přidá seznamy do obsahu automaticky
\hypersetup{
    colorlinks=true,
    linkcolor=blue,
    filecolor=magenta,      
    urlcolor=cyan,
    pdftitle={Dokumentace LOTR Systému},
    pdfpagemode=FullScreen,
}

% --- Nastavení pro výpisy kódu ---
\definecolor{codegreen}{rgb}{0,0.6,0}
\definecolor{codegray}{rgb}{0.5,0.5,0.5}
\definecolor{codepurple}{rgb}{0.58,0,0.82}
\definecolor{backcolour}{rgb}{0.95,0.95,0.92}

\lstdefinestyle{mystyle}{
    backgroundcolor=\color{backcolour},   
    commentstyle=\color{codegreen},
    keywordstyle=\color{magenta},
    numberstyle=\tiny\color{codegray},
    stringstyle=\color{codepurple},
    basicstyle=\ttfamily\footnotesize,
    breakatwhitespace=false,         
    breaklines=true,                 
    captionpos=b,                    
    keepspaces=true,                 
    numbers=left,                    
    numbersep=5pt,                  
    showspaces=false,                
    showstringspaces=false,
    showtabs=false,                  
    tabsize=2
}
\lstset{style=mystyle}

% --- Hloubka číslování a obsahu ---
\setcounter{secnumdepth}{3}
\setcounter{tocdepth}{2}


% --- Informace o dokumentu ---
\title{Technická dokumentace \\ GPS Tracking Systém ``LOTR''}
\author{Tým LOTR}
\date{\today}


% --- Začátek dokumentu ---
\begin{document}

% --- Titulní strana a obsah ---
\maketitle

% --- Prohlášení a poděkování ---
\cleardoublepage
\pagenumbering{roman}

\chapter*{Prohlášení}
\addcontentsline{toc}{chapter}{Prohlášení}
Prohlašujeme, že tento dokument byl vypracován samostatně a všechny použité zdroje jsou řádně citovány.

\vspace*{2cm}
\noindent V Praze dne \today \hfill \textit{Tým LOTR}

\chapter*{Poděkování}
\addcontentsline{toc}{chapter}{Poděkování}
Děkujeme všem, kteří se podíleli na tomto projektu.

\begin{abstract}
\addcontentsline{toc}{chapter}{Abstrakt}
Tento dokument poskytuje komplexní technický přehled systému pro sledování GPS zařízení ``LOTR''. Popisuje architekturu backendu, hardwarových komponent a mobilní aplikace.
\end{abstract}

	ableofcontents
\listoffigures
\listoftables
\lstlistoflistings

\cleardoublepage
\pagenumbering{arabic}

% --- Hlavní část dokumentu ---

\part{Analytická část}
% Zdroj(y): README.md, Poznámky k vývoji projektu.md
\chapter{Úvod}
\label{chap:uvod}

\section{Kontext a motivace}
Stručný popis problému sledování polohy, využití v praxi a motivace projektu LOTR.

\section{Cíle projektu}
\begin{itemize}
  \item Navrhnout a implementovat nízkoenergetický GPS tracker.
  \item Vytvořit backend (Node.js) s API a administračním rozhraním.
  \item Dodat mobilní aplikaci (Android) s funkcemi dohledu.
  \item Zajistit bezpečnost, škálovatelnost a spolehlivost řešení.
\end{itemize}

\section{Rozsah a omezení}
Co je a co není součástí tohoto projektu (např. iOS aplikace, pokročilá analytika atd.).

\section{Stakeholdeři a uživatelské role}
Přehled rolí: uživatel, administrátor, servisní technik.

\section{Pojmy a zkratky}
Seznam důležitých pojmů (GNSS, OTA, ERD, REST, JWT, ...). V případě potřeby rozšiřte v příloze Seznam zkratek.

% Zdroj(y): docs_apk/overview.md, docs_server/1-backend-overview.md
\chapter{Systémový přehled}
\label{chap:prehled}

\section{Komponenty systému}
Přehled hlavních komponent: HW tracker, backend server, databáze, web/administrace, mobilní aplikace.

\section{Use-cases a scénáře}
Typické scénáře: registrace zařízení, odesílání polohy, dohledové zóny, upozornění.

\section{Nefunkční požadavky}
Dostupnost, spolehlivost, škálovatelnost, bezpečnost, spotřeba energie.

% Zdroj(y): docs_apk/architecture.md, docs_server/1-backend-overview.md, docs_hw/hardware.md
\chapter{Architektura systému}
\label{chap:architektura}

\section{Celková architektura}
Vysokourovňový diagram komponent a toků dat.

\section{Komunikační rozhraní}
Protokoly a formáty (HTTP/HTTPS, REST, JSON, MQTT pokud relevantní).

\section{Diagramy}
Sekvenční diagramy pro klíčové toky: registrace, přenos polohy, OTA.


\part{Návrh a implementace}
% Zdroj(y): docs_server/1-backend-overview.md, docs_server/8-frontend.md, docs_server/9-administration.md
\chapter{Backend server (Node.js)}
\label{chap:server}

\section{Přehled a technologie}
Node.js, Express, nasazení, konfigurace prostředí.

\section{Struktura aplikace}
Adresářová struktura, hlavní moduly a jejich odpovědnosti.

\section{Konfigurace a nasazení}
Proměnné prostředí, build a běh, CI/CD (pokud existuje), monitoring.

% Zdroj(y): docs_server/2-database.md, docs_server/db_diagrams/*
\chapter{Databáze}
\label{chap:db}

\section{Model a ER diagram}
Entity, vztahy, schéma, indexy, migrační strategie.

\section{Integritní pravidla a výkonnost}
FK, validace, optimalizace dotazů.

% Zdroj(y): docs_server/3-api-and-routes.md, docs_server/4-authentication.md
\chapter{API a autentizace}
\label{chap:api-auth}

\section{Přehled REST API}
Hlavní routy, verze API, konvence, chybové kódy.

\section{Autentizace a autorizace}
JWT, správa relací, role a oprávnění, rate limiting.

% Zdroj(y): docs_hw/hardware.md, docs_hw/configuration.md, docs_hw/schemas/*
\chapter{Zařízení a hardware}
\label{chap:hardware}

\section{Popis komponent}
LilyGO T-Call, GNSS modul, napájení, baterie, antény.

\section{Schéma zapojení a konstrukce}
Elektrické schéma, mechanická konstrukce, pouzdro.

\section{Spotřeba a provozní profily}
Měření, optimalizace, doporučení.

% Zdroj(y): docs_hw/firmware.md, docs_hw/ota.md, docs_hw/configuration.md, docs_hw/data-format.md
\chapter{Firmware}
\label{chap:firmware}

\section{Architektura firmware}
Moduly, stavy, režimy spánku a probouzení.

\section{Konfigurace a OTA}
Konfigurační parametry, OTA aktualizace, servisní režim.

\section{Datové formáty a protokoly}
Struktura odesílaných dat, validace, zabezpečení přenosu.

% Zdroj(y): docs_apk/overview.md, docs_apk/architecture.md, docs_apk/ui.md, docs_apk/services.md, docs_apk/config.md, docs_apk/data.md, docs_apk/setup.md
\chapter{Mobilní aplikace (Android)}
\label{chap:apk}

\section{Architektura a použité knihovny}
Architektonický vzor, použitá SDK, knihovny, minimální verze Androidu.

\section{Uživatelské rozhraní}
Hlavní obrazovky, navigace, stavy, přístupnost.

\section{Služby a integrace}
Sběr polohy, komunikace se serverem, notifikace.

\section{Konfigurace a nasazení}
Build, signing, publikace (Google Play), prostředí.

% Zdroj(y): docs_server/7-gps-data-processing.md, docs_hw/data-format.md
\chapter{Zpracování GPS dat}
\label{chap:gps-data}

\section{Pipeline zpracování}
Příjem dat, validace, agregace, ukládání, notifikace.

\section{Datové modely}
Struktury pro body, trasy, geofencing.

\section{Výkon a optimalizace}
Dávkové zpracování, fronty (pokud jsou), indexy.

% Zdroj(y): docs_server/8-frontend.md, docs_server/9-administration.md
\chapter{Frontend a administrace}
\label{chap:frontend-admin}

\section{Webové rozhraní}
Mapy, vizualizace, filtrování, historie.

\section{Administrace a správa uživatelů}
Role, oprávnění, správa zařízení, auditní logy.


\part{Kvalita, provoz a bezpečnost}
% Zdroj(y): naplnit dle praxe projektu
\chapter{Testování a kvalita}
\label{chap:testovani}

\section{Unit a integrační testy}
Testovací strategie na úrovni serveru, firmware, aplikace.

\section{E2E testy a akceptační kritéria}
Scénáře, metriky úspěchu, automatizace.

\section{Měření výkonu a spolehlivosti}
Load testy, sledování dostupnosti, SLO/SLI.

% Zdroj(y): docs_apk/setup.md, docs_server/1-backend-overview.md
\chapter{Nasazení a provoz}
\label{chap:nasazeni}

\section{Provozní prostředí}
Dev/test/prod, závislosti, konfigurace.

\section{Build a release proces}
Versioning, changelog, release artefakty.

\section{Monitoring a logování}
Zdraví služeb, metriky, alarmy.

% Zdroj(y): konsolidovat z API/auth a HW/firmware
\chapter{Bezpečnost a soukromí}
\label{chap:bezpecnost}

\section{Bezpečnost komunikace a dat}
TLS, šifrování, ochrana proti útokům.

\section{Řízení přístupu a audit}
Role, zásady hesel, logování.

\section{Ochrana osobních údajů}
Zásady GDPR, uchovávání dat, anonymizace.


\part{Závěr}
\chapter{Závěr a další rozvoj}
\label{chap:zaver}

\section{Shrnutí výsledků}
Hlavní dosažené cíle a přínosy.

\section{Možnosti dalšího rozvoje}
Technické i produktové směry, dluhy a priorizace.


% --- Přílohy ---
\appendix
% Zdroj(y): docs_server/3-api-and-routes.md
\chapter{Příklady API volání}
\label{app:api-examples}

Ukázkové requesty a response v JSON. Vhodné vložit jako \texttt{listings} s typem \texttt{json}.

% Zdroj(y): docs_hw/schemas/*
\chapter{Schémata zapojení}
\label{app:schemata}

Vložte schémata zapojení (obr.) s popisky a legendou.

% Zdroj(y): docs_hw/troubleshooting.md
\chapter{Troubleshooting}
\label{app:troubleshooting}

Známé problémy a jejich řešení pro HW, firmware, server i aplikaci.


% --- Bibliografie ---
\cleardoublepage
\addcontentsline{toc}{chapter}{Literatura}
\bibliographystyle{czechiso} % Styl citací podle normy
%\bibliography{zdroje} % Název vašeho .bib souboru

\end{document}
